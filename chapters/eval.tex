\chapter{\ifproject%
\ifcpe การทดลองและผลลัพธ์\else Experimentation and Results\fi
\else%
\ifcpe การประเมินระบบ\else System Evaluation\fi
\fi}


\section{ระบบต้องทำงานได้ทั้งหมดดังนี้}
\subsection{การกรอกใบสมัคร}
\begin{itemize}
    \item การกรอกใบสมัครจะต้องครบถ้วนสมบูรณ์ตามเงื่อนไขที่กำหนดไว้
    \begin{itemize}
        \item วิธีการทดสอบคือ ต้องกรอกชื่อจริง ชื่อเล่น วันเดือนปีเกิด ไม่เช่นนั้น จะไม่สามารถดำเนินการต่อไปในขั้นตอนถัดไปได้ จำเป็นต้องกรอกให้ครบถ้วนก่อนเสมอ
    \end{itemize}
\end{itemize}
\subsection{การเช็คชื่อ}
\begin{itemize}
    \item ต้องสามารถเช็คชื่อในแต่ละวัน และ เช็คชื่อย้อนหลังได้
    \begin{itemize}
        \item ขั้นตอนการทดสอบการเช็คชื่อ  ก็คือ  ลองเช็คชื่อเด็กในแต่ละวัน  ถ้าเช็คว่ามาก็จะมีสัญลักษณ์เช็คถูกขึ้นมา ใน column ของวันที่เช็ค และ ในแถวของเด็กคนนั้น  หากเด็กคนนั้นไม่มา ก็จะมีสัญลักษณ์ขีดแดทขึ้นบนช่องนั้น
        \item ขั้นตอนการทดสอบการเช็คชื่อย้อนหลัง  ก็คือ  เมื่อต้องการแก้ไขให้เด็กคนหนึ่งที่มาเรียนสายมากๆ  จนทำให้คุณครูลืมเช็คชื่อ  จึงต้องทำการเช็คชื่อย้อนหลังในวันถัดๆไป  เพื่อแก้ไขให้เด็กที่มาสายคนนั้นในหน้าแสดงผลให้สัญลักษณ์ขีดแดท กลายเป็น สัญลักษณ์เช็คถูก  โดยจะตรวจสอบด้วยการเลือกวันที่ต้องการเช็คย้อนหลังแล้วทำการเช็คชื่อ  ถ้าหากระบบทำงานได้ปกติ  จะต้องได้ผลลัพธ์แบบที่กล่าวไว้ข้างต้น
    \end{itemize}
\end{itemize}
\subsection{การตรวจสุภาพประจำวัน}
\begin{itemize}
    \item สามารถเช็คสุขภาพประจำวันได้ครบถ้วนตามที่ระบุไว้ และ สามารถเช็คย้อนหลังได้
    \begin{itemize}
        \item ขั้นตอนการทดสอบ คือ เมื่อมีการเช็คของ ของเด็กแต่ละคน  เมื่อเช็คว่าเอามาใน column ของวันที่เช็ค ในแถวของเด็กนั้น จะมีสัญลักษณ์ เช็คถูกแสดงในช่องนั้น  ถ้าหากไม่ได้เอามาก็จะแสดงสัญลักษณ์ ขีดแดทที่ช่องนั้น
    \end{itemize}
\end{itemize}
\subsection{การตรวจอุปกรณ์ประจำวัน}
\begin{itemize}
    \item สามารถตรวจเช็คอุปกรณ์ประจำวันได้ครบถ้วนตามที่ระบุไว้และสามารถเช็คย้อนหลังได้
    \begin{itemize}
        \item ขั้นตอนการทดสอบคือ เมื่อตรวจสอบอุปกรณ์ของเด็กแต่ละคนจนเสร็จแล้ว  เราสามารถเข้าไปแก้ไขได้โดย  เมื่อแก้ไขเสร็จแล้ว  
        หน้าแสดงผลก็  ควรเปลี่ยนไปตามที่แก้ไขด้วยเช่นกัน        
    \end{itemize}
\end{itemize}
\subsection{แสดงประวัติส่วนตัว}
\begin{itemize}
    \item สามารถแสดงข้อมูลได้ครบถ้วน
    \begin{itemize}
        \item ขั้นตอนการทดสอบคือ ตรวจเช็คข้อมูลที่  แสดงผลดูว่าข้อมูลนั้นๆ ไม่ขาดตกบกพร่อง และ สามารถเรียกดูรายละเอียดต่างๆได้
        และ สามารถแก้ไขข้อมูลที่มีการเปลี่ยนของเด็กได้ เช่น น้ำหนัก  ส่วนสูง  ห้องเรียน         
    \end{itemize}
\end{itemize}
\subsection{แสดงสิ้นค้าในคลัง}
\begin{itemize}
    \item สามารถแสดงสินค้าทั้งหมดในคลังได้ และ สามารถแก้ไขจํานวนสิ้นค้าในคลังได้
    \begin{itemize}
        \item ขั้นตอนการทดสอบคือ  เมื่อมีสินค้าในคลังที่ถูกสั่งซื้อไปแล้ว จำนวนสิ้นค้าในคลังควรจะลดลง และ  ถ้ามีสิ้นค้าเข้ามาเพิ่มสินค้าในคลังก็ควรจะเพิ่มเช่นกัน  อีกทั้งเราสามารถลด และ เพิ่มจำนวนของสินค้าได้ตามต้องการ 	
    \end{itemize}
\end{itemize}
\subsection{การเก็บประวัติการจ่ายเงิน}
\begin{itemize}
    \item สามารถเก็บประวัติการจ่ายเงินได้ และ สามารถเรียกข้อมูลมาดูได้
    \begin{itemize}
        \item ขั้นตอนการทดสอบคือ เมื่อมีการจ่ายเงินค่าเทอมเราสามารถเก็บประวัติการจ่ายเงินนั้นแล้ว  นำมาแสดงว่าบุคคลนั้นจ่ายเงินไปรึยังโดยแสดงผ่านหน้าบัญชี ( Payment page )
    \end{itemize}
\end{itemize}


