\chapter{\ifcpe ทฤษฎีที่เกี่ยวข้อง\else Background Knowledge and Theory\fi}


\section{ด้านการออกแบบ (Design)}
จากการศึกษาค้นคว้า พบว่าตัวโครงงานที่เรากำลังจะพัฒนามีการเปลี่ยนแปลงของ requirements อยู่บ่อยครั้ง เช่น สัปดาห์ก่อนลูกค้าต้องการให้แค่เช็คชื่อในวันปัจจุบันได้เท่านั้น  แต่พอสัปดาห์นี้ลูกค้าต้องการเช็คชื่อย้อนหลังด้วยทางผู้พัฒนาจึงจะนำเอาแนวคิด design patterns \cite{designPatterns} เข้ามาทำงานช่วยในการทำโครงงานของเรา เนื่องจากแนวคิดนี้จะเข้ามาช่วยในการจัดการกับ requirements ที่มีการเปลี่ยนอยู่บ่อยครั้ง ให้มีความยืดหยุ่นมากขึ้น
ซึ่งหลักการ design patterns คือ แนวคิดที่ใช้ในกันในการแก้ไขปัญหาที่เกิดขึ้นอยู่บ่อยๆ ในการออกแบบ software และแนวคิดเหล่านี้ไม่เป็นรูปแบบที่ตายตัว แต่เป็นการอธิบายผ่านแนวคิดหรือโครงสร้างที่จะถูกนำไปประยุกต์ใช้ตามสถานการณ์ต่างๆ  
โดยจะผู้พัฒนาจะนำแนวคิดแบบ agile \cite{agile}  ซึ่งเป็นแนวคิดประเภทหนึ่งของ design patterns เข้ามาประยุกต์ใช้กับตัวโครงงาน agile คือ กระบวนการทำงานที่มุ่งเน้นการสื่อสาร ลดขั้นตอนการทำกับเอกสารลง แล้วนำมาพัฒนาชิ้นงานให้เสร็จเร็วยิ่งขึ้น  ซึ่งตัว agile นี้มีความน่าสนใจในส่วนของกระบวนการในการทำงานที่มีความโดดเด่นในเรื่องของความยืดหยุ่น  เนื่องจากวิธีดังกล่าวนี้มีการแจกแจงงานออกเป็นส่วนๆ  แล้วนำมาพัฒนากันเป็นทีละส่วน  
ซึ่งจุดเด่นของ agile นี้คือ การรองรับการเปลี่ยนแปลงของ requirements จากลูกค้า มากกว่าการทำตามแบบแผนที่วางไว้  เนื่องจากเมื่อทางผู้พัฒนามีการส่งตัว demo ไปยังลูกค้า  พบว่าสิ่งที่ต้องการลูกค้าแต่แรกไม่ตรงตามที่คิดไว้  จึงทำให้เกิดการเพิ่มหรือแก้ไข requirements กันอยู่บ่อยครั้ง
\CIreply{ย่อหน้านี้พูดวนมาก ลองสรุปดูว่าประเด็นที่เราพยายามจะสื่ออะไร แล้วลองเขียนใหม่}

โดยจุดเด่นในเรื่อง\CI{นี้}{เรื่องไหน}จะตอบโจทย์ในของความรวดเร็วในการส่งงาน หรือ ตัว demo ให้กับลูกค้า และสามารถนำ feedback จากลูกค้ามาแก้ไขได้ เพื่อให้ตัวชิ้นงานมีความสมบูรณ์ยิ่งขึ้น 
 จากประโยชน์ทั้งหมดที่กล่าวมาข้างต้น ทางผู้พัฒนาจึงเลือกใช้ agile นำในการนำมาออกแบบ flow ในการทำงานต่างๆ  โดยทางผู้พัฒนาจะนำกระบวนการทำงานแบบ scrum \cite{srcum} มาใช้ในการวางแผนการทำงาน 
ซึ่ง\CI{เจ้าตัวนี้}{?}เป็นหนึ่งในแนวคิดของ agile ซึ่งมีประสิทธิภาพใน\CI{การจัดการ}{จัดการอะไร} กับปริมาณงานให้มีความสมดุล  โดยการแบ่งงานออกเป็นส่วนๆ แล้วแยกกันมาพัฒนา  ถ้าส่วน\CI{นึง}{ภาษาพูด}เสร็จแล้วจะมีการส่งไป test  หากผ่านขั้นตอนนี้จะเคลื่อนไปทำงานในส่วนถัดไป  หากไม่ผ่านก็จะวนกลับไปในขั้นตอนเดิมเพื่อกลับพัฒนาให้ดีขึ้นอีกครั้ง ซึ่งกระบวนที่กล่าวไปข้างต้นนี้  จะช่วยให้เราสามารถโฟกัสกับตัวชิ้นงานได้ดียิ่งขึ้น เนื่องจากมีการเจาะลงไป ทำเป็นทีละส่วน ทำให้สามารถที่จะพัฒนาได้อย่างมีประสิทธิภาพกว่าการทำเป็นส่วนใหญ่ทีเดียว
\CIreply{เพิ่ม diagram อธิบายขั้นตอนของ scrum ให้เห็นภาพมากขึ้น}

ในส่วนของการออกแบบฐานข้อมูล ทางผู้พัฒนาเลือกใช้ในตัวของ MongoDB \cite{MongoDB} ซึ่งฐานข้อมูลตัวนี้จะเป็นฐานข้อมูลแบบไม่มีความสัมพันธ์ (NoSQL) 
เหตุผลที่เลือกใช้ฐานข้อมูลตัวนี้เนื่องจาก MongoDB มี services และ tools หลายตัวที่ช่วยอำนวยความสะดวกในการจัดการกับข้อมูลเป็นอย่างดีเช่น MongoDB Atlas, MongoDB Compass 
ข้อดีอีกอย่างคือ NoSQL สามารถ scalable เพื่อรองรับการขยายฐานข้อมูลในอนาคต ในส่วนโครงสร้างของระบบทางผู้พัฒนาเลือกใช้เป็นแบบ client-server \cite{Client-Server} \CI{แต่ในอนาคต}{แล้วได้ทำไหม}  ผู้พัฒนามีความคิดจะนำระบบโครงสร้างแบบ microservices \cite{Microservice} มาใช้ ถ้าเกิดการพัฒนาไปจนถึงครบทุกฟีเจอร์ที่ทางผู้พัฒนาตั้งไว้  
ตัว microservices จะเข้ามาช่วยในการจัดสรรแบ่งส่วนเป็นหมวดหมู่เช่น  ส่วนของเช็คชื่อ  ส่วนของเช็ค stock  ส่วนของ payment  โดยทางผู้พัฒนาคิดว่าถ้าเกิดจะใช้โครงสร้างแบบ Microservices ทางผู้พัฒนาคิดจะใช้ \CI{Kubernets}{check spelling} ( k8s ) มาช่วยในการทำ devops เพื่อให้สามารถจัดการกับตัว project ได้สะดวกยิ่งขึ้น\CIreply{แล้วได้ทำไหม} 
ตัว Kubernets เป็นแพลตฟอร์ม open-source ที่จะช่วยให้การทำงานต่างๆ ที่เกี่ยวข้องกับ \CI{Linux containers}{คืออะไร} สามารถทำได้โดยอัตโนมัติและลดกระบวนการติดตั้งหรือขยายแอปพลิเคชันที่รันบน containers ที่ทางผู้พัฒนาต้องลงมือทำด้วยตนเองให้เหลือน้อยที่สุด 
	ในส่วนของการออกแบบ UX/UI ทางเราเลือกใช้เครื่องมือคือ Adobe XD \cite{AdobeXD} เนื่องจากโปรแกรมนี้มีให้ใช้งานได้ฟรีและใช้งานง่าย โดยไม่ต้องเสียเวลาไปเรียนรู้ สำหรับตัวโปรแกรม Adobe XD เป็น software ที่ใช้สำหรับวาดแบบหรือออกแบบหน้า UI ที่ใช้งานง่าย 
\CIreply{check ย่อหน้า ว่าควรขึ้นใหม่ตรงไหน}
\CIreply{รู้สึกว่าร่ายไปเรื่อยมาก แต่ละย่อหน้า ควรพูดทีละประเด็น}
  
  

\section{ด้านเทคนิค (Technical)}
ในส่วนของทฤษฏีหรือองค์ความรู้ในด้านนี้จะหนักไปในเรื่องของการทำ
web application ในส่วนของหน้าบ้าน (frontend) ทางผู้พัฒนาเลือกใช้ในตัวของ
React \cite{React} ซึ่งเป็น Library ของ JavaScript ตัวหนึ่งที่ได้รับความนิยมเป็นอยากมาก เหตุผลที่เลือกใช้เนื่องจากตัว library React นี้ถูกสร้างจากทาง Facebook 
จึงมี community อย่างหนาแน่น  เมื่อพบเจอปัญหาที่ไม่สามารถแก้ไขด้วยตนเอง สามารถไปค้นหาจากแหล่ง community เหล่านี้ได้ ยกตัวอย่างเช่น StackOverflow 
ยิ่งไปกว่านั้นเหตุผลที่เลือกใช้ React ก็คือ Redux \cite{Redux}\CI{,  tools}{ทำไมใช้ comma} ตัวนี้ช่วยอำนวยความสะดวกในเรื่องของการจัดการกับ states ต่างๆ  ภายใน project ให้สามารถเรียกใช้ states ต่างๆได้อย่างเป็นระบบระเบียบและการใช้งาน React ยังมีรูปแบบการเขียน
ได้หลากหลายรูปแบบ อย่างเช่น class components, hooks ซึ่งเราสามารถเลือกใช้ได้ตามความเหมาะสมของงานเพื่อทำให้ชิ้นงานมีประสิทธิภาพสูงสุด
\CIreply{ย่อหน้า?}
ส่วนหลังบ้าน (backend) ทางผู้พัฒนาเลือกใช้ Node.js \cite{NodeJs}
และ Express.js \cite{Express} ซึ่ง Node.js ใช้สำหรับรันตัว JavaScript ไว้ทำงานบนฝั่ง Backend ส่วน Express.js คือ  framework บน Node.js 
ซึ่งตัว Express มี features ต่างๆ  ที่จะช่วยให้ทำเว็บได้สะดวกขึ้น เช่น การทำ routing (เพื่อให้ client สามารถส่งข้อมูลไปยัง server ได้) การจัดการ request และ response เป็นต้น  ทำให้เราสามารถพัฒนาเว็บโดยใช้ Node.js ได้สะดวกและรวดเร็วยิ่งขึ้นตัว
เนื่องจากทางตัวโครงงานฝั่ง frontend เราก็ใช้ภาษา JavaScript แล้ว backend เราก็เลยเลือกใช้ Express.js  เนื่องจากทางผู้พัฒนาจะได้ไม่ต้องเสียเวลาไปศึกษาภาษาอื่นๆ เช่น .NET Framework, Java, Python 
ส่วนประโยชน์ของเจ้าตัว Node.js ก็คือในเรื่องของความเร็ว performance และ ยังง่ายต่อการศึกษาเรียนรู้ทำให้ไม่เสียเวลาในการเรียนรู้มาก


\section{\ifcpe%
ความรู้ตามหลักสูตรซึ่งถูกนำมาใช้หรือบูรณาการในโครงงาน
\else%
ISNE knowledge used, applied, or integrated in this project
\fi
}

ความรู้ตามหลักสูตรที่ถูกนำมาใช้ในโครงงานมีดังต่อไปนี้
\begin{itemize}
  \item ความรู้จากวิชา Computer Programming ช่วยทำให้เข้าใจการเขียนโค้ดพื้นฐานและสามารถนำไปประยุกต์ในการเขียนภาษาอื่นๆได้
  \item ความรู้จากวิชา Basic Computer Lab ช่วยทำให้เข้าใจพื้นฐานการเขียน web application เบื้องต้นและเข้าใจคำสั่งพื้นฐานในการเขียน Linux CLI
  \item ความรู้จากวิชา Database ช่วยทำให้เข้าใจการออกแบบฐานและรู้คำสั่ง query ข้อมูลพื้นฐาน
  \item ความรู้จากวิชา Software Engineering ช่วยทำให้เข้าใจวงจรการทำงานให้เกิดเป็น software ชิ้นหนึ่งและเข้าใจว่ากระบวนการหรือแนวคิดในการสร้าง software มีอยู่หลากหลายวิธี สามารถเลือกใช้ได้ตามความเหมาะสมของชิ้นงาน 
  
\end{itemize}


\section{\ifcpe%
ความรู้นอกหลักสูตรซึ่งถูกนำมาใช้หรือบูรณาการในโครงงาน
\else%
Extracurricular knowledge used, applied, or integrated in this project
\fi
}

ความรู้นอกหลักสูตรที่ถูกนำมาใช้ในโครงงานมีดังต่อไปนี้
\begin{itemize}
  \item ความรู้เรื่องการเขียน React นำมาเพื่อเขียนหน้า browser ให้ทาง user ใช้งาน
  \item ความรู้เรื่องการใช้ Redux นำมาเพื่อใช้ในการจัดการกับ states ต่างๆในโค้ดส่วน frontend
  \item ความรู้ในการนำ Express.js เข้ามาช่วยในการเขียน routing, จัดการ requestและresponse
\end{itemize}
