\chapter{\ifcpe ทฤษฎีที่เกี่ยวข้อง\else Background Knowledge and Theory\fi}


\section{ด้านการออกแบบ( Design )}
จากการศึกษาค้นคว้าจากทางทีมผู้พัฒนาพบว่าตัวโครงงานที่เรากำลังจะพัฒนามีการเปลี่ยนแปลงของ Requirement อยู่บ่อยครั้ง  ทางผู้พัฒนาจึงจะนำเอาแนวคิด Design Patterns \cite{designPatterns} เข้ามาทำงานช่วยในการทำโครงงานของเรา 
ซึ่งหลักการ Design Patterns คือ แนวคิดที่ใช้ในกันในการแก้ไขปัญหาที่เกิดขึ้นอยู่บ่อยๆ ในการออกแบบ Software และแนวคิดเหล่านี้ไม่เป็นรูปแบบที่ตายตัวง แต่เป็นการอธิบายผ่านแนวคิดหรือโครงสร้างที่จะถูกนำไปประยุกต์ใช้ตามสถานการณ์ต่างๆ  
โดยจะผู้พัฒนาจะนำแนวคิดแบบ Agile \cite{agile}  ซึ่งเป็นแนวคิดประเภทหนึ่งของ Design Patterns เข้ามาประยุกต์ใช้กับตัวโครงงาน   ซึ่งตัว Agile นี้มีความน่าสนใจในส่วนของกระบวนการในการทำงานที่มีความโดดเด่นในเรื่องของความยืดหยุ่น  เนื่องจากวิธีดังกล่าวนี้มีการแจกแจงงานออกเป็นส่วนๆ  แล้วนำมาพัฒนากันเป็นทีละส่วน  
ซึ่งจุดเด่นของ Agile นี้คือ การรองรับการเปลี่ยนแปลงของ Requirement จากลูกค้า มากกว่าการทำตามแบบแผนที่วางไว้  โดยจุดเด่นในเรื่องนี้จะตอบโจทย์ในของความรวดเร็วในการส่งงานหรือตัว Demo ให้กับลูกค้าและ สามารถนำ Feedback จากลูกค้ามาแก้ไขได้เพื่อให้ตัวชิ้นงานมีความสมบูรณ์ยิ่งขึ้น 
 จากประโยชน์ทั้งหมดที่กล่าวมาข้างต้นทางทีมผู้พัฒนาจึงเลือกใช้ Agile นำในการนำมาออกแบบ Flow ในการทำงานต่างๆ  โดยทางผู้พัฒนาจะนำกระบวนการทำงานแบบ Scrum \cite{srcum} มาใช้ในการวางแผนการทำงาน 
ซึ่งเจ้าตัวนี้เป็นหนึ่งในแนวคิดของ Agile ซึ่งมีประสิทธิภาพในการจัดการกับปริมาณงานให้มีความสมดุลโดยการแบ่งงานออกเป็นส่วนๆ แล้วแยกกันมาพัฒนา  ถ้าส่วนนึงเสร็จแล้วจะมีการส่งไป Test  หากผ่านขั้นนี้ก็จะเคลื่อนไปทำงานในส่วนถัดไป  หากไม่ผ่านก็จะวนกลับไปในขั้นตอนเดิมเพื่อกลับพัฒนาให้ดีขึ้นอีกครั้ง  
ในส่วนของการออกแบบฐานข้อมูล ทางผู้พัฒนาเลือกใช้ในตัวของ Mongodb \cite{MongoDB} ซึ่งฐานข้อมูลตัวนี้จะเป็นฐานข้อมูลแบบไม่มีความสัมพันธ์ ( NoSQL ) 
เหตุผลที่เลือกใช้ฐานข้อมูลตัวนี้เนื่องจาก Mongodb มี Services และ Tools หลายตัวที่ช่วยอำนวยความสะดวกในการจัดการกับข้อมูลเป็นอย่างดีอย่างเช่น MongoDB Atlas, MongoDB Compass 
ข้อดีอีกอย่างคือ Nosql มีสามารถในการ Scalable เพื่อรองรับการขยายฐานข้อมูลในภายภาคหน้า  ในส่วนโครงสร้างของระบบทางผู้พัฒนาเลือกใช้เป็นแบบ Client-Server \cite{Client-Server} แต่ในภายภาคหน้า  ผู้พัฒนามีความคิดที่จะนำเอาระบบโครงสร้างแบบ Microservices \cite{Microservice}มาใช้ถ้าเกิดพัฒนาไปจนถึงจุดที่มีฟีตเจอร์เกินกว่าที่ทางทีมงานแพลนไว้  
ตัว Microservices จะเข้ามาช่วยในการจัดสรรแบ่งส่วนเป็นหมวดหมู่  ยกตัวอย่างเช่น  ส่วนของเช็คชื่อ  ส่วนของเช็ค Stock  ส่วนของ Payment  โดยทางผู้พัฒนาคิดว่าถ้าเกิดจะใช้โครงสร้างแบบ Microservices ทางผู้พัฒนาคิดว่าจะใช้ตัว Kubernets ( k8s ) มาช่วยในการทำ Devops เพื่อให้สามารถจัดการกับตัว Project ได้สะดวกยิ่งขึ้น 
ตัว Kubernets เป็นแพลตฟอร์ม Open-source ที่จะช่วยให้การทำงานต่างๆ ที่เกี่ยวข้องกับ Linux Container สามารถทำได้โดยอัตโนมัติ และลดกระบวนการติดตั้งหรือขยายแอปพลิเคชันที่รันบน Container ที่ทางผู้พัฒนาต้องลงมือทำด้วยตนเองให้เหลือน้อยที่สุด 
	ในส่วนของการออกแบบ UX/UI ทางเราเลือกใช้เครื่องมือคือ Adobe XD \cite{AdobeXD} เนื่องจากเจ้าตัวโปรแกรมนี้มีให้ใช้งานได้ฟรีและยังใช้งานง่ายโดยไม่ต้องเสียเวลาไปเรียนรู้ สำหรับตัวโปรแกรม Adobe XD เป็น Software ที่ใช้สำหรับวาดแบบหรือออกแบบหน้า UI ที่ใช้งานง่าย 
  
  
  

\section{ด้านเทคนิค ( Technical )}
ในส่วนของทฤษฏีหรือองค์ความรู้ในด้านนี้จะหนักไปในเรื่องของการทำ
Web-Application ในส่วนของหน้าบ้าน ( Frontend ) ทางผู้พัฒนาเลือกใช้ในตัวของ
React \cite{React} ซึ่งเป็น Library ของ Javascript ตัวหนึ่งที่ได้รับความนิยมเป็นอยากมาก เหตุผลที่เลือกใช้เนื่องจากตัว Library React นี้ถูกสร้างจากทาง Facebook 
จึงมี Community อย่างหนาแน่น  เมื่อพบเจอปัญหาที่ไม่สามารถแก้ไขด้วยตนเองได้ก็สามารถไปค้นหาจากแหล่ง Community เหล่านี้ได้ ยกตัวอย่างเช่น Stackoverflow 
และอีกเหตุผลที่เลือกใช้ React ก็คือ เจ้าตัว Redux \cite{Redux} ,  Tools ตัวนี้ช่วยอำนวยความสะดวกในเรื่องของการจัดการกับ States ต่างๆ  ภายใน Project ให้สามารถเรียกใช้ States ต่างๆได้อย่างเป็นระบบระเบียบ และการใช้งาน React ยังมีรูปแบบการเขียน
ได้หลากหลายรูปแบบ อย่างเช่น Class Component , Hook ซึ่งเราสามารถเลือกใช้ได้ตามความเหมาะสมของงานเพื่อทำให้ชิ้นงานมีประสิทธิภาพสูงสุด
ส่วนหลังบ้าน ( Backend ) ทางผู้พัฒนาเลือกใช้ตัว Node.js \cite{NodeJs} และ Express.js \cite{Express} ซึ่งตัว Node.js คือตัวใช้สำหรับรันตัว Javascript ไว้ทำงานบนฝั่ง Backend ได้  ส่วน Express.js คือ  framework บน Node.js 
ซึ่งตัว Express จะมี Features ต่างๆ  ที่จะช่วยให้ทำเว็บได้สะดวกขึ้น ยกตัวอย่างเช่น การทำ Routing, Middleware การจัดการ Request และ Response เป็นต้น  ทำให้เราสามารถพัฒนาเว็บโดยใช้ Node.js ได้สะดวกและรวดเร็วยิ่งขึ้นตัว
เนื่องจากทางตัวโครงงานฝั่ง Frontend เราก็ใช้ภาษา Javascript แล้ว Backend เราก็เลยเลือกใช้ Express.js  เนื่องจากทางผู้พัฒนาจะได้ไม่ต้องเสียเวลาไปศึกษาภาษาอื่นๆ เช่น  .NET Framework , Java , Python 
ส่วนประโยชน์ของเจ้าตัว Nodejs ก็คือในเรื่องของความเร็ว และ Performance ยังไม่หมดแค่นี้ Nodejs ยังง่ายต่อการศึกษาเรียนรู้ทำให้ไม่เสียเวลาในการเรียนรู้มากนะ 




% \subsection{Subsection heading goes here}

% Subsection 1 text

% \subsubsection{Subsubsection 1 heading goes here}
% Subsubsection 1 text

% \subsubsection{Subsubsection 2 heading goes here}
Subsubsection 2 text

\section{ ด้านธุรกิจ ( Business )
}
ในส่วนของด้านธุรกิจตอนนี้ทางเรายังไม่ค่อยได้วางแผนสักเท่าไหร่ แต่ว่าทางผู้พัฒนาก็คิดว่าถ้าหากทางทีมงานพัฒนาตัวweb-applicationจนถึงขั้นสามารถนำไปใช้งานได้ในทุกๆfeaturesในระดับที่ทางผู้พัฒนาตั้งไว้ก็จะมีการนำไปเสนอให้แก่ทางnurseryหลายๆแห่งเพื่อที่จะทำการขายตัวแอปตัวนี้แก่พวกเขาเหล่านี้ โดยทางเราจะหารายได้จากการเพิ่มฟีดเจอร์ต่างๆไม่ก็การปรับปรุงแก้ไขระบบให้มีความเสถียรมากยิ่งขึ้น คอยดูแลระบบเก่าให้ยังสามารถใช้งานได้ต่อไปเรื่อยๆ 	เมื่อทางnurseryต้องการขยายระบบทางทีมงานก็ต้องพร้อมที่จะscalableระบบอยู่เสมอ ซึ่งในส่วนนี้ก็ถือว่าเป็นรายได้ส่วนหนึ่งที่เราจะได้รับมาจากทางลูกค้า


\section{\ifcpe%
ความรู้ตามหลักสูตรซึ่งถูกนำมาใช้หรือบูรณาการในโครงงาน
\else%
ISNE knowledge used, applied, or integrated in this project
\fi
}

ความรู้ตามหลักสูตรที่ถูกนำมาใช้ในโครงงานมีดังต่อไปนี้
\begin{itemize}
  \item ความรู้จากวิชา Computer Programming ช่วยทำให้เข้าใจการเขียนโค้ดพื้นฐาน และ สามารถนำไปประยุกต์ในการเขียนภาษาอื่นๆได้
  \item ความรู้จากวิชา Basic Computer Lab ช่วยทำให้เข้าใจพื้นฐานการเขียน Web-Application เบื้องต้น และ เข้าใจคำสั่งพื้นฐานในการเขียน Linux CLI
  \item ความรู้จากวิชา Database ช่วยทำให้เข้าใจการออกแบบฐาน และ รู้คำสั่ง Query ข้อมูลพื้นฐาน
  \item ความรู้จากวิชา Software Engineering ช่วยทำให้เข้าใจวงจรการทำงานให้เกิดเป็น Software ชิ้นหนึ่ง และ เข้าใจว่ากระบวนการหรือแนวคิดในการสร้าง Software มีอยู่หลากหลายวิธี สามารถเลือกใช้ได้ตามความ
  เหมาะสมของชิ้นงาน 
  
\end{itemize}


\section{\ifcpe%
ความรู้นอกหลักสูตรซึ่งถูกนำมาใช้หรือบูรณาการในโครงงาน
\else%
Extracurricular knowledge used, applied, or integrated in this project
\fi
}

ความรู้นอกหลักสูตรที่ถูกนำมาใช้ในโครงงานมีดังต่อไปนี้
\begin{itemize}
  \item ความรู้เรื่องการเขียน React นำมาเพื่อเขียนหน้า Browser ให้ทาง User ใช้งาน
  \item ความรู้เรื่องการใช้ Redux นำมาเพื่อใช้ในการจัดการกับ States ต่างๆในโค้ดส่วน Frontend
  \item ความรู้ในการนำ Express.js เข้ามาช่วยในการเขียน Routing , จัดการกับ Middleware , จัดการ Request และ Response
\end{itemize}
