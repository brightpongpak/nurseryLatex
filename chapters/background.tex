\chapter{\ifcpe ทฤษฎีที่เกี่ยวข้อง\else Background Knowledge and Theory\fi}


\section{ด้านการออกแบบ( Design )}
จากการศึกษาค้นคว้าจากทีมผู้พัฒนาพบว่าแนวคิด Design patterns ในหัวข้อเรื่องของ Agile มีความน่าสนใจเกี่ยวกับกระบวนการในการทำงานที่มีความโดดเด่นในเรื่องของความยืดหยุ่นเนื่องจากวิธีดังกล่าวนี้มีการแจกแจงงานออกเป็นส่วนๆแล้วนำมาพัฒนากันเป็นส่วนๆ โดยจุดเด่นของagileนี้คือ การรองรับการเปลี่ยนแปลงของ Requirementจากลูกค้า มากกว่าการทำตามแบบแผนที่วางไว้ ซึ่งจุดเด่นในเรื่องนี้จะตอบโจทย์ในของความรวดเร็วในการส่งงานให้กับลูกค้าและการแก้ไขfeedbackจากลูกค้า จากประโยชน์ทั้งหมดที่กล่าวมาข้างต้นทางทีมผู้พัฒนาจึงเลือกใช้ agile นำในการนำมาออกแบบflowในการทำงานต่างๆ โดยทางผู้พัฒนาจะนำแนวคิดเรื่องscrum มาใช้ในการวางแผนการทำงาน ซึ่งเจ้าตัวนี้เป็นหนึ่งในแนวคิดของagileซึ่งมีประสิทธิภาพในการจัดการกับปริมาณงานให้มีความสมดุลโดยการแบ่งงานออกเป็นส่วนๆแล้วแยกกันมาพัฒนาถ้าส่วนนึงเสร็จแล้วจะมีการส่งไปtestหากผ่านขั้นนี้ก็จะเคลื่อนไปทำงานในส่วนถัดไปหากไม่ผ่านก็จะวนกลับไปในขั้นตอนที่พัฒนาอีกครั้ง  
ในส่วนของฐานข้อมูลทางผู้พัฒนาเลือกใช้ในตัวของ mongodb ซึ่งฐานข้อมูลตัวนี้จะเป็นฐานข้อมูลแบบไม่มีความสัมพันธ์ ( NoSQL ) เหตุผลที่เลือกใช้ฐานข้อมูลตัวนี้เนื่องจาก mongodb มีservicesและtoolsหลายตัวที่ช่วยอำนวยความสะดวกในการจัดการกับข้อมูลเป็นอย่างดีอย่างเช่น MongoDB Atlas, MongoDB Compass และข้อดีอีกอย่างคือ nosql มีสามารถในการ scalable เพื่อรองรับการขยายฐานข้อมูลในภายภาคหน้า ในส่วนโครงสร้างของระบบที่ทางผู้พัฒนาเลือกใช้เป็นแบบ client-server แต่ในภายภาคหน้าผู้พัฒนามีความคิดที่จะนำเอาระบบโครงสร้างแบบ microservices มาใช้ถ้าเกิดพัฒนาไปจนถึงจุดที่มีฟีตเจอร์เกินกว่าที่ทางทีมงานแพลนไว้ ตัวmicroservicesจะเข้ามาช่วยในการจัดสรรแบ่งส่วนเป็นหมวดหมู่ ยกตัวอย่างเช่น ส่วนของเช็คชื่อ ส่วนของเช็คstock ส่วนของpayment โดยทางผู้พัฒนาคิดว่าถ้าเกิดจะใช้โครงสร้างแบบmicroservices ทางผู้พัฒนาคิดว่าจะใช้ตัวkubernets(k8s)มาช่วยในการทำdevopsเพื่อให้สามารถจัดการกับตัวprojectได้สะดวกยิ่งขึ้น
	ในส่วนของการออกแบบUX/UIทางเราเลือกใช้toolsคือ adobe xdเนื่องจากเจ้าตัวโปรแกรมนี้มีให้ใช้งานได้ฟรีและยังใช้งานง่ายโดยไม่ต้องเสียเวลาไปเรียนรู้
  เทคโนโลยีและแนวคิดที่ใช้

  Design patterns 
  คือแบบแผนหรือแนวคิดที่ใช้ในการแก้ไขปัญหาที่เกิดขึ้นบ่อยๆ ในการออกแบบซอฟต์แวร์ แบบแผนและแนวคิดเหล่านี้ไม่ใช่รูปแบบตายตัวที่จะถูกนำไปใช้โดยตรง แต่เป็นการอธิบายผ่านแนวคิดหรือโครงสร้างที่จะถูกนำไปประยุกต์ใช้ในสถานการณ์ต่างๆ 
  ดีไซน์แพตเทิร์นจะแสดงความสัมพันธ์ต่อกันระหว่างคลาสหรืออ็อบเจกต์ต่างๆ โดยไม่จำเพาะเจาะจงการนำไปใช้งานในขั้นสุดท้าย ขั้นตอนวิธีไม่จัดเป็นดีไซน์แพตเทิร์นเพราะเป็นการแก้ปัญหาในทางการประมวลผลมากกว่าในทางการออกแบบ
  ประโยชน์ของDesign patterns
  ดีไซน์แพตเทิร์นช่วยทำให้กระบวนการพัฒนาโปรแกรมรวดเร็วขึ้นเนื่องจากเป็นตัวอย่างที่ผ่านการพิสูจน์ทดลองมาแล้ว การออกแบบซอฟต์แวร์ที่ดีต้องเตรียมการสำหรับปัญหาที่อาจจะไม่พบจนกว่าจะเริ่มนำไปใช้งาน การใช้ดีไซน์แพตเทิร์นช่วยป้องกันปัญหาเล็กๆน้อยๆที่อาจจะลุกลามใหญ่โต ทั้งยังทำให้การทำความเข้าใจโค้ดง่ายขึ้นในหมู่ผู้ร่วมงานในทีมที่คุ้นเคยกับดีไซน์แพตเทิร์น
  
  https://th.wikipedia.org/wiki/%E0%B8%94%E0%B8%B5%E0%B9%84%E0%B8%8B%E0%B8%99%E0%B9%8C%E0%B9%81%E0%B8%9E%E0%B8%95%E0%B9%80%E0%B8%97%E0%B8%B4%E0%B8%A3%E0%B9%8C%E0%B8%99
  Agile
    แนวคิดแบบอไจล์ คือ การปรับปรุงกระบวนการให้ทำงานได้เร็วขึ้นและมีประสิทธิภาพมากยิ่งขึ้น แต่ใช่ว่าอไจล์จะเหมาะสำหรับทุกองค์กร แต่ละองค์กรย่อมมีวัฒนธรรมการทำงานที่แตกต่างกันออกไป ดังนั้นการนำแนวคิดแบบอไจน์มาใช้จึงไม่มีแบบสำเร็จตายตัว แต่ละองค์กรจะต้องหาวิธีมาปรับใช้ให้เหมาะกับวัฒนธรรมขององค์กรเอง
    Agile คือ แนวคิดในการทำงาน และไม่จำกัดว่าแค่ต้องนำไปใช้กับการพัฒนาซอฟต์แวร์เท่านั้น แต่อไจล์ให้ความสำคัญในเรื่อง คน การสื่อสาร และ แนวทางที่จะนำไปใช้พัฒนาสินค้าและบริการ ขององค์กรให้ได้อย่างรวดเร็ว เพื่อให้สินค้าและบริการเหล่านั้นสามารถตอบสนองต่อความต้องการของผู้ใช้งานรวมถึงผู้บริโภคอยู่เสมอ
    อไจล์เป็นกระบวนการที่ช่วยลดการทำงานที่เป็นขั้นตอนและงานด้านการทำเอกสารลง แต่จะไปมุ่งเน้นในเรื่องการสื่อสารของทีมมากขึ้น เพื่อให้เกิดการพัฒนาสินค้าและบริการใหม่ๆ ได้รวดเร็วขึ้น แล้วจึงนำสิ่งที่ได้ไปให้ผู้ใช้กลุ่มตัวอย่าง (Target group) ทดสอบใช้งานจริง จากนั้นจึงรวมรวมผลทดสอบมาประเมินดูอีกครั้ง เพื่อใช้เป็นแนวทางในการแก้ไขปรับปรุงสินค้าและบริการนั้นๆ ให้ดีขึ้นทีละนิด ด้วยแนวทางนี้จะทำให้องค์กรสามารถพัฒนาสินค้าและบริการได้อย่างรวดเร็วและตอบโจทย์ผู้ใช้งานได้มากขึ้นอย่างสม่ำเสมอนั่นเอง
  
  หลักการทำงานแบบอไจล์ประกอบด้วย
  มีการทำงานแบบ Cross-functional team คือการนำคนที่มาจากหลากหลายสายงานที่มักมีความคิดเห็นต่างกัน มาทำงานร่วมกันอยู่ในทีมเดียวกัน สิ่งนี้จะส่งผลให้ทีมสามารถทำความเข้าใจกับรายละเอียดของงานได้ดีและง่ายขึ้นแล้วยังส่งผลถึงเรื่องการประสานงานกับฝ่ายงานต่างๆ ให้มีความคล่องตัวมากยิ่งขึ้น
  ทีมมีอำนาจในการในการตัดสินใจและกำหนดทิศทางของโครงการมากขึ้น ส่วนใหญ่คนที่ได้รับมอบหมายให้ปฏิบัติงานภายใน Agile squad จะได้รับอำนาจในการตัดสินใจที่มากพอเพื่อไม่ให้โครงการต้องผ่านกระบวนการการขออนุมัติจากองค์กรที่มักจะกินเวลานาน หมายความว่า Product Owner จะต้องมีอำนาจตัดสินใจได้ด้วยตัวเอง เพื่อให้เกิดความสะดวกรวดเร็วที่สุด
  ใช้บุคลากรที่ทำงานเพื่อโครงการนี้โดยเฉพาะ (Dedicated resources) มีการจัดตั้งคนที่รับผิดชอบงานในแต่ละส่วน เพื่อให้โฟกัสภายใน Scope  of work ของโครงการที่ได้รับมอบหมายมา
  แบ่งเฟสงานให้เป็นโครงการเล็กๆ กำหนดเป้าหมายที่ใช้ระยะเวลาสั้นๆ และต้องส่งมอบชิ้นงานเป็นโครงการเล็กๆ เมื่อประเมินผลแล้วพบว่าอยู่ในทิศทางที่ดีจึงค่อยๆต่อยอดเพิ่มไปเรื่อยๆ ซึ่งหากพบข้อผิดพลาดหรือจำเป็นต้องมีการเปลี่ยนแปลงใดๆ ก็จะปรับเปลี่ยนแผนการทำงานให้เหมาะสมในแต่ละรอบไป มักเรียกวิธีการนี้ว่า Sprint
  ทุกคนสามารถรับรู้สถานะของโครงการได้อย่างชัดเจน ทุกคนจะต้องสื่อสารและรับรู้ปัญหาที่เกิดขึ้นภายในโครงการ รวมทั้งรายงานความคืบหน้าของโครงการให้ทั้งทีมได้รับรู้ เพื่อทำให้เกิดความชัดเจนและสามารถวัดผลได้
  เกิดการเรียนรู้อยู่เสมอ การเรียนรู้ข้อผิดพลาดและข้อดีได้อย่างรวดเร็วถือเป็นสิ่งจำเป็น เนื่องจากการทำงานเป็นรอบสั้นๆ ทำให้เกิดการเรียนรู้ข้อผิดพลาดที่พบจากครั้งก่อนๆ และสามารถหาข้อบกพร่องตลอดจนข้อดีในการทำงานได้อย่างรวดเร็ว
  https://www.tnt.co.th/news/162-agile-framework-working-principle-for-modern-company
  
  
  
  Scrum
    Scrum (สกรัม) คือการนำเอาแนวคิดในการทำงานแบบ Agile (อไจล์) มาปฏิบัติตามขั้นตอนของสกรัม เพื่อระบุปัญหาที่มีความซับซ้อน เปลี่ยนแปลงบ่อย ให้สามารถส่งมอบผลิตภัณฑ์ที่ตอบสนองต่อการเปลี่ยนแปลงที่เกิดขึ้นได้อย่างรวดเร็ว
  ทฤษฏีสกรัม (Scrum Theory)
  Scrum เน้นการนำความรู้จากประสบการณ์เฉพาะที่เคยลงมือทำจริง(Empiricism) มาพัฒนาการดำเนินงานในปัจจุบันให้ดียิ่งขึ้น ประกอบด้วย 3 ส่วน
  ความโปร่งใส (Transparency) คือทีมจะต้องเห็นภาพชัดเจนและเข้าใจตรงกัน มาตรฐานเดียวกันและไม่ตีความหมายต่างกัน เช่น นิยามของคำว่างานเสร็จ หมายถึง การผลิตเสร็จ หรือ ผลิตและทดสอบเสร็จ หรือ ได้รับการเซ็นรับรอง หรือ ส่งมอบให้ผู้ใช้แล้ว ต้องนิยามและตกลงให้ตรงกันก่อน
  การตรวจสอบ (Inspection) คือการนำผลลัพธ์จากการดำเนินกิจกรรมต่างๆของสกรัม(Scrum Artifact) มาตรวจสอบและวัดผลว่าบรรลุตามวัตถุประสงค์ที่ตั้งไว้รึเปล่า
  การปรับเปลี่ยน (Adoption) คือหากผลลัพธ์ไม่เป็นไปตามที่ตั้งไว้ จะต้องปรับเปลี่ยนวิธีการดำเนินงาน หรือจำนวนทรัพยากรที่ใช้ เพื่อให้บรรลุผลตามที่ตั้งไว้หรือใกล้เคียงให้ได้มากที่สุด
  https://medium.com/fastwork-engineering/scrum-คืออะไร-เริ่มใช้งานอย่างไร-2483e761a47e
  Mongodb
  MongoDB เป็น open-source document database โดยเป็นฐานข้อมูลแบบ NoSQL คือไม่มี relation (ความสัมพันธ์) ของตารางแบบ SQL ทั่วไป แต่จะเก็บข้อมูลเป็นแบบ JSON (JavaScript Object Notation) แทน การบันทึกข้อมูลทุกๆ record ใน MongoDB เราจะเรียกมันว่า Document ซึ่งจะเก็บค่าเป็น key และ value 
  https://devahoy.com/blog/2015/08/getting-started-with-mongodb/
  
  Client-Server
  เครือข่ายแบบ Client/Server เป็นรูปแบบหนึ่งของเครือข่ายแบบ server-based โดยจะมีคอมพิวเตอร์เครื่องหลักหนึ่งเครื่องเป็น เซิร์ฟเวอร์ ซึ่งจะไม่ได้ทำหน้าที่ประมวลผลทั้งหมดให้เครื่องลูกข่ายหรือเครื่องไคลเอนต์ (client) แต่เซิร์ฟเวอร์จะทำหน้าที่เสมือนเป็นที่เก็บข้อมูลระยะไกล และประมวลผลบางอย่างให้กับเครื่องไคลเอนต์เท่านั้น เช่น ประมวลผลคำสั่งในการดึงข้อมูลจากเซิร์ฟเวอร์ฐานข้อมูล (database server) เป็นต้น
  https://www.snc.co.th/Article/Detail/111211
  Microservices
  Microservice ถือเป็นเทคนิคหนึ่ง ในการทำ Software development ซึ่งก็จัดอยู่ใน Service-Oriented Architecture (SOA) ที่จะสร้าง Application ของคุณแบบไม่ผูกมัดกันจนเกินไป โดยที่จะให้การทำงานของ Application มีขนาดไม่ใหญ่ (lightweight)
  ประโยชน์ของการแยก Application ให้มาอยู่ในรูปแบบของ Service ต่าง ๆ ทำให้ Application มีขนาดเล็กลง มีความเข้าใจในตัว Service มากขึ้น Development หรือ Testing ง่ายขึ้น และมีความยืดหยุ่นมากขึ้นด้วย
  https://medium.com/@iamgique/หลัก-6-ประการของ-microservice-f13f4b82a169
  เครื่องมือที่ใช้
  Adobe XD
  คือ โปรแกรมที่ถูกสร้างมาเพื่อใช้ในการออกแบบหน้าUX/UIของโปแกรมที่เรากำลังจะพัฒนาขึ้นด้วย ตัวโปรแกรมนี้มีฟังก์ชันพิเศษอย่างเช่นการแสดงหน้าprototype เพื่อให้ลูกค้าสามารถลองใช้งานinterfaceต่างๆภายในโปรแกรมได้โดยที่ทางผู้พัฒนายังไม่ต้องลงไปเขียนโปรแกรมเองเลย
  https://www.grappik.com/inside-adobe-xd
  

\section{ด้านเทคนิค ( Technical )}
ในส่วนของทฤษฏีหรือองค์ความรู้ในด้านนี้จะหนักไปในเรื่องของการทำ
web-application ในส่วนของหน้าบ้าน( frontend )ทางผู้พัฒนาเลือกใช้ในตัวของ
react ซึ่งเป็น library ของ javascript ตัวหนึ่งที่ได้รับความนิยมเป็นอยากมาก เหตุผลที่เลือกใช้เนื่องจากตัวlibrary reactนี้ถูกสร้างจากทางfacebook 
จึงมีcommunityอย่างหนาแน่น เมื่อพบเจอปัญหาที่ไม่สามารถแก้ไขด้วยตนเองได้ก็สามารถไปค้นหาจากแหล่งcommunityเหล่านี้ได้ ยกตัวอย่างเช่น stackoverflow 
และอีกเหตุผลที่เลือกใช้ react ก็คือ เจ้าตัว redux toolsตัวนี้ช่วยอำนวยความสะดวกในเรื่องของการจัดการstatesต่างๆภายในproject ให้สามารถเรียกใช้statesต่างๆได้อย่างเป็นระบบระเบียบ และการเรียนreactยังมีรูปแบบการเขียน
ได้หลากหลายรูปแบบ อย่างเช่น class component , hook ซึ่งเราสามารถเลือกใช้ได้ตามความเหมาะสมของงานเพื่อทำให้เนื้อหามีประสิทธิภาพสูงสุด
ส่วนหลังบ้าน( backend )ทางผู้พัฒนาเลือกใช้ตัว nodejs เนื่องจากทางfrontendเราก็ใช้ภาษาjavascriptแล้วbackendเราก็เลยเลือกใช้nodejsเนื่องจากทางผู้พัฒนาจะได้ไม่ต้องเสียเวลาไปศึกษาภาษาอื่นๆ เช่น Java, python และประโยชน์ของเจ้าตัวnodejsก็คือในเรื่องของความเร็วและperformance ยังไม่หมดแค่นี้nodejsยังง่ายต่อการศึกษาเรียนรู้ทำให้ไม่เสียเวลาในการเรียนรู้มากนะ 

ภาษาโปรแกรมที่ใช้
React
React คือ JavaScript Library ที่ทีม Facebook เป็นคนพัฒนาขึ้นมา และเปิดให้คนทั่วไปนำมาใช้งานได้ฟรี ซึ่งเว็บไซต์ในปัจจุบันของ Facebook.comก็ใช้Reactอยู่เช่นกันครับ
    สรุปคือ คอนเซปต์ที่เราต้องรู้เพื่อเขียน React หลัก ๆ มีแค่ 3 Concept เท่านั้นเอง
Component – คือส่วนต่างๆ ในเว็บไซค์ของเรา จะมองเป็น Component ต่างๆ
State – ข้อมูลที่อยู่ใน Component แต่ละตัว
Props – ข้อมูลที่ถูกส่งต่อจาก Component หนึ่งไปอีกComponentหนึ่ง
https://www.designil.com/react-คืออะไร.html
Redux
คือ State Management (ตัวจัดการ State) และมันก็ไม่จำเป็นว่าจะต้องใช้กับ React เท่านั้นนะครับ มันเป็น Concept เท่านั้น ฉะนั้นมันสามารถประยุกต์ใช้ได้หมด เช่น Angular หรือ Vue
กฎหลักสามข้อของreduxมีดังนี้
	1.Single Source of Truth กล่าวคือเราจะเก็บstateไว้ที่storeเพียงที่เดียวให้storeเป็นตัวกลางในการจัดเก็บstates
	2.State is read-only กล่าวคือ เราจะสามารถแก้ไขstatesผ่านทางactionทางเดียวเท่านั้น
	3.Changes are made with pure functions การเปลี่ยนแปลง state ต้องเป็น Pure function เท่านั้น ก็คือ ภายใน reducers ของเราสามารถเปลี่ยน state ได้ แต่ไม่ใช่การแก้ไข state เดิม เป็นการส่งค่า state ใหม่กลับมาแทน
https://devahoy.com/blog/2018/07/introduction-to-redux/

Node.js
คือ JavaScript Runtime ที่ถูกสร้างด้วย Chrome’V8 JavaScript Engine ทำหน้าที่อยู่ฝั่ง Backend ทำตัวเป็น Web Server
https://medium.com/@settawatjanpuk/https-medium-com-settawatjanpuk-beginner-node-js-970383cc6e3a
Express.js
เป็น web application framework บน Node.js ที่ได้รับความนิยมมากตัวหนึ่ง
ซึ่งตัว Express เนี่ยจะมีฟีจเจอร์ต่างๆที่จะช่วย.ให้ทำเว็บได้สะดวกขึ้น เช่น การทำ routing, middleware การจัดการ request และ response เป็นต้น ทำให้เราสามารถพัฒนาเว็บโดยใช้ Node.js ได้สะดวกและรวดเร็วยิ่งขึ้น
https://medium.com/@aofleejay/สร้าง-restful-api-ด้วย-express-express-101-ee37cc4952b4




เครื่องมือที่ใช้

VSCode
		ตัวeditorที่ถูกพัฒนาจากทาง microsoft เพื่ออำนวยความสะดวกในการเขียนโค้ดหรือพัฒนาโปรแกรม โดยทางโปรแกรมนี้ก็มีส่วนเสริมมากมายให้เลือกใช้ตามประเภทของงานที่ผู้ใช้กำลังพัฒนาอยู่เพื่อเพิ่มประสิทธิภาพในการพัฒนาและลดความผิดพลาดของชิ้นงานเพิ่มขึ้น




% \subsection{Subsection heading goes here}

% Subsection 1 text

% \subsubsection{Subsubsection 1 heading goes here}
% Subsubsection 1 text

% \subsubsection{Subsubsection 2 heading goes here}
Subsubsection 2 text

\section{ ด้านธุรกิจ ( Business )
}
ในส่วนของด้านธุรกิจตอนนี้ทางเรายังไม่ค่อยได้วางแผนสักเท่าไหร่ แต่ว่าทางผู้พัฒนาก็คิดว่าถ้าหากทางทีมงานพัฒนาตัวweb-applicationจนถึงขั้นสามารถนำไปใช้งานได้ในทุกๆfeaturesในระดับที่ทางผู้พัฒนาตั้งไว้ก็จะมีการนำไปเสนอให้แก่ทางnurseryหลายๆแห่งเพื่อที่จะทำการขายตัวแอปตัวนี้แก่พวกเขาเหล่านี้ โดยทางเราจะหารายได้จากการเพิ่มฟีดเจอร์ต่างๆไม่ก็การปรับปรุงแก้ไขระบบให้มีความเสถียรมากยิ่งขึ้น คอยดูแลระบบเก่าให้ยังสามารถใช้งานได้ต่อไปเรื่อยๆ 	เมื่อทางnurseryต้องการขยายระบบทางทีมงานก็ต้องพร้อมที่จะscalableระบบอยู่เสมอ ซึ่งในส่วนนี้ก็ถือว่าเป็นรายได้ส่วนหนึ่งที่เราจะได้รับมาจากทางลูกค้า


\section{About using figures in your report}

% define a command that produces some filler text, the lorem ipsum.
\newcommand{\loremipsum}{
  \textit{Lorem ipsum dolor sit amet, consectetur adipisicing elit, sed do
  eiusmod tempor incididunt ut labore et dolore magna aliqua. Ut enim ad
  minim veniam, quis nostrud exercitation ullamco laboris nisi ut
  aliquip ex ea commodo consequat. Duis aute irure dolor in
  reprehenderit in voluptate velit esse cillum dolore eu fugiat nulla
  pariatur. Excepteur sint occaecat cupidatat non proident, sunt in
  culpa qui officia deserunt mollit anim id est laborum.}\par}

\begin{figure}
  \centering

  \fbox{
     \parbox{.6\textwidth}{\loremipsum}
  }

  % To include an image in the figure, say myimage.pdf, you could use
  % the following code. Look up the documentation for the package
  % graphicx for more information.
  % \includegraphics[width=\textwidth]{myimage}

  \caption[Sample figure]{This figure is a sample containing \gls{lorem ipsum},
  showing you how you can include figures and glossary in your report.
  You can specify a shorter caption that will appear in the List of Figures.}
  \label{fig:sample-figure}
\end{figure}

Using \verb.\label. and \verb.\ref. commands allows us to refer to
figures easily. If we can refer to Figures
\ref{fig:walrus} and \ref{fig:sample-figure} by name in the {\LaTeX}
source code, then we will not need to update the code that refers to it
even if the placement or ordering of the figures changes.

\loremipsum\loremipsum

% This code demonstrates how to get a landscape table or figure. It
% uses the package lscape to turn everything but the page number into
% landscape orientation. Everything should be included within an
% \afterpage{ .... } to avoid causing a page break too early.
\afterpage{
  \begin{landscape}
  \begin{table}
    \caption{Sample landscape table}
    \label{tab:sample-table}

    \centering

    \begin{tabular}{c||c|c}
        Year & A & B \\
        \hline\hline
        1989 & 12 & 23 \\
        1990 & 4 & 9 \\
        1991 & 3 & 6 \\
    \end{tabular}
  \end{table}
  \end{landscape}
}

\loremipsum\loremipsum\loremipsum

\section{Overfull hbox}

When the \verb.semifinal. option is passed to the \verb.cpecmu. document class,
any line that is longer than the line width, i.e., an overfull hbox, will be
highlighted with a black solid rule:
\begin{center}
\begin{minipage}{2em}
juxtaposition
\end{minipage}
\end{center}

\section{\ifcpe%
ความรู้ตามหลักสูตรซึ่งถูกนำมาใช้หรือบูรณาการในโครงงาน
\else%
ISNE knowledge used, applied, or integrated in this project
\fi
}

อธิบายถึงความรู้ และแนวทางการนำความรู้ต่างๆ ที่ได้เรียนตามหลักสูตร ซึ่งถูกนำมาใช้ในโครงงาน

\section{\ifcpe%
ความรู้นอกหลักสูตรซึ่งถูกนำมาใช้หรือบูรณาการในโครงงาน
\else%
Extracurricular knowledge used, applied, or integrated in this project
\fi
}

อธิบายถึงความรู้ต่างๆ ที่เรียนรู้ด้วยตนเอง และแนวทางการนำความรู้เหล่านั้นมาใช้ในโครงงาน
