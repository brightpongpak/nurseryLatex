\chapter{\ifproject%
\ifcpe โครงสร้างและขั้นตอนการทำงาน\else Project Structure and Methodology\fi
\else%
\ifcpe โครงสร้างของโครงงาน\else Project Structure\fi
\fi
}



\section{โครงสร้างของระบบ}



\subsection{Database Design}

ประเภทของฐานข้อมูลที่ใช้เป็นแบบ NoSQL
(รูปที่~\ref{fig:DatabaseDiagram}) แสดงองค์ประกอบต่างๆ ของฐานข้อมูล ซึ่งมีรายละเอียดดังต่อไปนี้
%
  \begin{itemize}
    \item childs เก็บข้อมูลที่เกี่ยวข้องกับเด็กทั้งหมด เช่น เก็บวันที่สมัคร วันที่เข้าเรียน ชื่อ ชื่อเล่น วันเดือนปีเกิด เป็นต้น  
    โดยที่ \_id จะมีการเชื่อมกับ payment, gadgets, healths, enrollment, address, guardians, customer, childStatic, childDynamic, medical, document, child\_info 
    ในส่วนของการเชื่อมความสัมพันธ์นี้ ทำเพื่อให้การเรียกใช้ข้อมูลของเด็กในส่วนของข้อมูลอื่น เช่น การเช็คชื่อ การชำระเงินค่าเรียน การเช็คของ เกิดขึ้นได้โดยสะดวกยิ่งขึ้น
    \item childStatic เก็บข้อมูลเด็กที่ไม่เปลี่ยนแปลง เช่น เชื้อชาติ ศาสนา วันที่สมัคร
    \item childDynamic เก็บข้อมูลเด็กที่มีการเปลี่ยนแปลง เช่น น้ำหนัก ส่วนสูง รูปถ่าย
    \item child\_info เก็บข้อมูลพฤติกรรมต่างๆ ของเด็ก
    \item info\_category เก็บข้อมูลชื่อประเภทหัวเรื่องต่างๆ
    \item info\_item เก็บข้อมูลหยิบย่อยต่างๆ
    \item info\_display\_order เก็บข้อมูลสำหรับทำหน้าแสดงผลแบบ dynamic
    \item customer เก็บข้อมูลลูกค้าสำหรับใช้ซื้อของจากทางเนอสเซอรี่
    \item medical เก็บข้อมูลทางการแพทย์ของเด็ก
    \item document เก็บข้อมูลเอกสารในการสมัครของเด็ก
    \item guardians เก็บข้อมูลส่วนตัวของผู้ปกครองเด็ก เช่น ชื่อ ทำอาชีพอะไร อีเมล เบอร์โทร มีความสัมพันธ์อะไรกับเด็ก เป็นต้น
    \item address เก็บข้อมูลที่อยู่ของเด็ก เช่น บ้านเลขที่ ถนน ตำบล อำเภอ จังหวัด รหัสไปรษณีย์ 
    \item users เก็บ email password สำหรับเข้าสู่ ระบบจัดการ nursery 
    \item rooms เก็บข้อมูลห้องเรียนว่าห้องนี้ชื่อห้องว่าอะไร ไว้ให้ enrollment เรียกใช้ผ่าน ObjectId
    \item enrollment เก็บข้อมูลว่าเด็กเริ่มเรียนห้องนี้วันไหน มีการย้ายออกจากห้องนี้วันไหน
    \item attendances เก็บข้อมูลการเช็คชื่อทั้งหมด วันที่เช็ค สถานะว่ามาหรือไม่มา
    \item gadgets เก็บข้อมูลการเช็คอุปกรณ์ของเด็กทั้งหมด มีการเก็บ ข้อมูลเด็ก วัน ลิสต์ของที่ต้องเช็ค ก็จะมีสถานะว่าเอามาหรือไม่เอามา
    \item healths เก็บข้อมูลการเช็คสุขภาพของเด็กทั้งหมด มีการเก็บ ข้อมูลเด็ก วัน ลิสต์สุขภาพที่ต้องเช็ค ก็จะมีสถานะว่าใช่หรือไม่ใช่
    \item stock เก็บข้อมูลของใช้ทั้งหมดของเด็กใน nursery ว่าของชิ้นนี้ชื่ออะไร \CI{ไซต์}{check spelling}อะไร รหัสสินค้าคืออะไร
    \item show\_stock เก็บข้อมูลว่าของต่างๆ มีจำนวน\CI{เท่าไหร่}{ภาษาพูด} ราคาเท่าไหร่
    \item history\_stock เก็บข้อมูลการเพิ่มลดของต่างๆ \CI{จำนวน}{จำนวนอะไร} แก้ไขวันไหน โดยใคร
    \item payment เก็บข้อมูลประวัติการโอนเงินค่าเล่าเรียนของผู้ปกครองเด็ก มีการเก็บ ข้อมูลเด็ก วันที่ส่งสลิปโอนเงิน ค่าเล่าเรียนหรือจำนวนเงินที่ชำระ รูปสลิปใบเสร็จ 
  \end{itemize}

\subsection{XD Design}

\begin{enumerate}
  \item RegisterPage คือ หน้าแบบฟอร์มในการลงทะเบียนเด็กในการเข้าเรียน nursery (รูปที่~\ref{fig:register}, \ref{fig:docForm})
  \item  ProfilePage คือ หน้าที่ใช้ในการจัดการกับประวัติส่วนตัวของเด็ก สามารถแสดงข้อมูลและแก้ไขข้อมูลของเด็กแต่ละคนได้ (รูปที่~\ref{fig:Profile}, \ref{fig:ChartPage}, \ref{fig:ProfileTwo}, \ref{fig:UpdateProfile})
  \item  Attendance คือ หน้าเช็คชื่อเด็กรายวัน (รูปที่~\ref{fig:Attendance}, \ref{fig:CheckAttendance})
  \item  Health คือ หน้าเช็คสุขภาพเด็กรายวัน (รูปที่~\ref{fig:Health}, \ref{fig:CheckHealth}) 
  \item  Gadget คือ หน้าเช็คของเด็กรายวัน (รูปที่~\ref{fig:Gadget}, \ref{fig:CheckGadget}) 
  \item  Stock คือ หน้าจัดการของใช้สำหรับเด็กทั้งหมดใน nursery (รูปที่~\ref{fig:Stock}, \ref{fig:CheckStock}, \ref{fig:editPrice}, \ref{fig:HistoryStock}) 
  \item  Payment คือ หน้าที่ใช้จัดการกับใบเสร็จ ประวัติการจ่ายเงินค่าเล่าเรียน (รูปที่~\ref{fig:Payment}, \ref{fig:CreatePayment}, \ref{fig:updatePayment}, \ref{fig:invoicePage}, \ref{fig:slipPage}) 
\end{enumerate}
\CIreply{อธิบาย feature ของแต่ละหน้าว่าทำอะไรได้บ้าง ผู้ใช้ต้องทำอะไร และระบบจะบริการอะไรให้  แต่ละรูปควรจะเขียนเป็น bullet point ได้ ทำให้หัวข้อนี้โดยรวมน่าจะยาวประมาณ 3 หน้า ไม่รวมรูป  หากคำอธิบายเริ่มยาว ให้ค่อยๆ ย้ายรูปมาแทรกในส่วนของ text ให้เหมาะสม}

\subsection{Architecture}

ระบบหลักๆที่ทางผู้พัฒนาใช้เป็นระบบแบบ client--server คือทางฝั่ง client จะใช้ React ในการพัฒนาตัวเว็บไซต์ โดยผู้ใช้จะมีการส่ง request มายังฝั่ง server เพื่อที่จะทำงานที่ผู้ใช้ต้องการ ซึ่งฝั่ง server 
จะพัฒนาผ่าน Express.js ทางฝั่งนี้ก็จะ query ข้อมูลจากทางฐานข้อมูลที่เก็บไว้บน MongoDB Atlas เมื่อได้รับข้อมูลแล้วจะทำการส่ง response กลับไปยัง client เพื่อให้ผู้ใช้สามารถทำงานในส่วนที่ต้องการได้

\subsection{APIs Docs}
\CIreply{อธิบายโดยคร่าวๆ ด้วย}
(รูปที่~\ref{fig:ApiDocs1}, \ref{fig:ApiDocs2})
\CIreply{หมายเลขรูปไม่เรียง ให้ย้าย figure ตามลำดับที่ควรจะเป็น หรือย้าย section text ตรงนี้}
\section{ขั้นตอนการดำเนินงาน}
\subsection{Discovery}
\begin{itemize}
  \item สำรวจและสอบถามปัญหาจาก stakeholder
  \item นำปัญหาต่างหรือ requirements มาวิเคราะห์
  \item สรุปผลแล้วนำ requirements ที่ได้จากการวิเคราะห์ไปทำต่อในขั้นตอนถัดไป
\end{itemize}

\subsection{Design}
\begin{itemize}
  \item ออกแบบหน้า UI/UX โดย Adobe XD
  \item ออกแบบฐานข้อมูลโดย Draw.io
\end{itemize}

\subsection{Develop}
\begin{itemize}
  \item สร้าง database
  \item เขียน APIs ตามที่ได้ design มาในขั้นตอนก่อนหน้า
  \item เขียน frontend แล้วทดสอบยิง APIs ไปยังฝั่ง Backend
  \item เชื่อมโค้ด frontend กับ backend ผ่าน APIs
\end{itemize}

\subsection{Testing}
\begin{itemize}
  \item ทดสอบระบบในแต่ละฟีเจอร์ว่าสามารถใช้งานได้ตามที่ต้องการหรือไม่
\end{itemize}

\begin{landscape}
  \begin{figure}
    \begin{center}
    \includegraphics[height=0.9\textheight]{images/NurseryDiagram.png}
    \end{center}
  \caption{แผนภาพแสดงรายละเอียดฐานข้อมูลของระบบ}
  \label{fig:DatabaseDiagram}
\end{figure}
\end{landscape}

\begin{figure}
  \begin{center}
    \includegraphics[width=\linewidth]{images/ApiDocOne.png}
  \end{center}
  \caption[ตารางแสดง API Document 1]{ApiDocs1}
  \label{fig:ApiDocs1}
\end{figure}
\CIreply{check \texttt{caption} syntax: วงเล็บเหลี่ยมสำหรับจข้อความในส่วนสารบัญ วงเล็บปีกกาสำหรับข้อความใต้รูปภาพ; ถ้าไม่ใช้วงเล็บเหลี่ยม default จะเป็นอันเดียวกับที่อยู่ในปีกกา}

\CIreply{เล็กเกิน มองไม่เห็น ใช้ landscape?}

\begin{figure}
  \begin{center}
    \includegraphics[width=\linewidth]{images/ApiDocTwo.png}
  \end{center}
  \caption[ตารางแสดง API Document 2]{ApiDocs2}
  \label{fig:ApiDocs2}
\end{figure}
\CIreply{เล็กเกิน มองไม่เห็น ใช้ landscape?}

\CIreply{screenshots ควรทำเป็น landscape เพื่อทำให้มองเห็น}
\begin{figure}
  \begin{center}
  \includegraphics[width=\linewidth]{images/RegisterForm.png}
  \end{center}
  \caption[หน้าลงทะเบียนเด็ก]{Register Form}
  \label{fig:register}
  \end{figure}

\begin{figure}
  \begin{center}
  \includegraphics[width=\linewidth]{images/DocForm.png}
  \end{center}
  \caption[หน้าลงทะเบียนเอกสาร]{DocForm}
  \label{fig:docForm}
  \end{figure}

\begin{figure}
  \begin{center}
  \includegraphics[width=\linewidth]{images/Profile.png}
  \end{center}
  \caption[หน้าโปรไฟล์]{Profile Page}
  \label{fig:Profile}
  \end{figure}

\begin{figure}
  \begin{center}
  \includegraphics[width=\linewidth]{images/ChartPage.png}
  \end{center}
  \caption[หน้าแสดงข้อมูลรายเดือนเด็ก]{Show Chart Page}
  \label{fig:ChartPage}
  \end{figure}
  
\begin{figure}
  \begin{center}
  \includegraphics[width=\linewidth]{images/ProfileInfo.png}
  \end{center}
  \caption[หน้าประวัติส่วนตัวเด็ก]{ProfileInfo Page}
  \label{fig:ProfileTwo}
  \end{figure}

\begin{figure}
  \begin{center}
  \includegraphics[width=\linewidth]{images/updateProfile.png}
  \end{center}
  \caption[หน้าแก้ไขข้อมูลเด็ก]{Update Profile Modal}
  \label{fig:UpdateProfile}
  \end{figure}
\begin{figure}
  \begin{center}
  \includegraphics[width=\linewidth]{images/Attendance.png}
  \end{center}
  \caption[หน้าแสดงการเข้าเรียนเด็ก]{Attendance Page}
  \label{fig:Attendance}
  \end{figure}

\begin{figure}
  \begin{center}
  \includegraphics[width=\linewidth]{images/checkAttendance.png}
  \end{center}
  \caption[หน้าเช็คชื่อเด็ก]{Attendance Checking Modal}
  \label{fig:CheckAttendance}
  \end{figure}

\begin{figure}
  \begin{center}
  \includegraphics[width=\linewidth]{images/Health.png}
  \end{center}
  \caption[หน้าแสดงข้อมูลสุขภาพเด็กรายวัน]{Health Page}
  \label{fig:Health}
  \end{figure}

\begin{figure}
  \begin{center}
  \includegraphics[width=\linewidth]{images/checkHealth.png}
  \end{center}
  \caption[หน้าเช็คสุขภาพเด็ก]{Health Checking Modal}
  \label{fig:CheckHealth}
  \end{figure}

\begin{figure}
  \begin{center}
  \includegraphics[width=\linewidth]{images/Gadget.png}
  \end{center}
  \caption[หน้าแสดงข้อมูลอุปกรณ์เด็กรายวัน]{Gadget Page}
  \label{fig:Gadget}
  \end{figure}

\begin{figure}
  \begin{center}
  \includegraphics[width=\linewidth]{images/checkGadget.png}
  \end{center}
  \caption[หน้าเช็คอุปกรณ์เด็ก]{Gadget Checking Modal}
  \label{fig:CheckGadget}
  \end{figure}

\begin{figure}
  \begin{center}
  \includegraphics[width=\linewidth]{images/Stock.png}
  \end{center}
  \caption[หน้าแสดงคลังสินค้า]{Stock Page}
  \label{fig:Stock}
  \end{figure}

\begin{figure}
  \begin{center}
  \includegraphics[width=\linewidth]{images/handleStock.png}
  \end{center}
  \caption[หน้าจัดการแก้ไขสต็อกของ]{Handle Stock Page}
  \label{fig:CheckStock}
  \end{figure}
\begin{figure}
  \begin{center}
  \includegraphics[width=\linewidth]{images/editPrice.png}
  \end{center}
  \caption[หน้าจัดการแก้ไขราคาสินค้า]{Edit Stock Price Page}
  \label{fig:editPrice}
  \end{figure}
\begin{figure}
  \begin{center}
  \includegraphics[width=\linewidth]{images/historyStock.png}
  \end{center}
  \caption[หน้าแสดงประวัติการเพิ่มลดของ]{Show History Stock Page}
  \label{fig:HistoryStock}
  \end{figure}
\begin{figure}
  \begin{center}
  \includegraphics[width=\linewidth]{images/Payment.png}
  \end{center}
  \caption[หน้าแสดงบัญชี]{Payment Page}
  \label{fig:Payment}

  \end{figure}
\begin{figure}
  \begin{center}
  \includegraphics[width=\linewidth]{images/CreatePayment.png}
  \end{center}
  \caption[หน้าเพิ่มรายการบัญชี]{CreatePayment}
  \label{fig:CreatePayment}
  \end{figure}
\CIreply{ใหญ่เกิน}

\begin{figure}
  \begin{center}
  \includegraphics[width=\linewidth]{images/UpdatePayment.png}
  \end{center}
  \caption[หน้าชำระค่าใช้จ่าย]{Handle Payment}
  \label{fig:updatePayment}
  \end{figure}
\begin{figure}
  \begin{center}
  \includegraphics[width=\linewidth]{images/invoicePage.png}
  \end{center}
  \caption[หน้าปริ้นใบแจ้งยอด]{Page for printing invoice}
  \label{fig:invoicePage}
  \end{figure}
\begin{figure}
  \begin{center}
  \includegraphics[width=\linewidth]{images/slipPage.png}
  \end{center}
  \caption[หน้าปริ้นใบเสร็จ]{Page for printing slip}
  \label{fig:slipPage}
  \end{figure}