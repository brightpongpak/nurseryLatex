\chapter{\ifcpe บทนำ\else Introduction\fi}

\section{\ifcpe ที่มาของโครงงาน\else Project rationale\fi}

ในปัจจุบันเทคโนโลยีได้เข้ามามีบทบาทในชีวิตประจําวันในทุกๆ ด้าน โดยในแต่ละด้านก็มีการนําเทคโนโลยีมาใช้ทดแทนแรงงานคน  เพื่ออํานวยความสะดวกในการทํางานของเจ้าหน้าที่ฝ่ายต่างๆ  ยกตัวอย่างเช่น แพลตฟอร์มลงทะเบียนสอบ GAT/PAT ออนไลน์ application ที่ใช้จัดการกับระบบต่างๆภายในบริษัทหรือองค์กร สำนักงานต่างๆ ระบบจัดการภายในโรงพยาบาล ระบบจองคิวรับคิวภายในร้านอาหาร เป็นต้น

แต่ว่าในภาคส่วนของทาง nursery ยังไม่ค่อยพบเห็นแพลตฟอร์มหรือระบบจัดการภายในแบบออนไลน์เท่าใดนัก
เนื่องจาก nursery เป็นองค์กรที่มีขนาดเล็ก ไม่ได้มีระบบการบริหารจัดการจากส่วนกลาง จึงทำให้ผู้พัฒนาหลายๆ กลุ่มมองไม่เห็นความสำคัญของปัญหาที่เกี่ยวกับการจัดการ nursery ได้แก่ การจัดเก็บข้อมูลต่างๆ รายวัน ซึ่งในปัจจุบันยังทำลงบนกระดาษอยู่ ส่งผลให้เกิดปัญหาความล่าช้าในกระบวนการรวบรวมและจัดเก็บข้อมูล
นอกจากนี้ การแก้ไขหรือเข้าถึงข้อมูลต่างๆ เช่น การตรวจสอบและแก้ไขประวัติเด็ก ย่อมเกิดความล่าช้า เนื่องจากจำเป็นต้องดำเนินการผ่านเจ้าหน้าที่ของ nursery ยิ่งไปกว่านั้น ยังมีเรื่องของความเสี่ยงในการที่เอกสารที่จัดเก็บไว้จะสูญหายหรือสูญเสียอีกด้วย

จากเหตุผลที่กล่าวมาข้างต้น ทางผู้จัดทําจึงได้นำแนวคิดนี้มาสร้างเป็น web application 
สําหรับจัดการภายในเนอสเซอรี่ เพื่อให้ครูและเจ้าหน้าที่ใน nursery สามารถตรวจสอบและแสดงข้อมูลของเด็กได้โดยง่ายและสะดวกรวดเร็วขึ้น และเพื่อลดการสูญหายของข้อมูลระหว่างการจัดส่งหรือจัดเก็บ ที่อาจเกิดขึ้นได้จากการใช้กระดาษ
นอกเหนือจากการจัดเก็บข้อมูลส่วนบุคคลของเด็กแต่ละคนแล้ว ระบบจัดการภายใน nursery ที่ผู้จัดทำได้วางแผนพัฒนานั้น จะสามารถรองรับการจัดการกิจวัตรประจำวันต่างๆ ภายใน nursery ได้แก่ การเช็คชื่อ เช็คของใช้ เช็คสุขภาพ
ซึ่งสามารถทำได้แบบ real-time ผ่านระบบออนไลน์
ในท้ายที่สุด ระบบจัดการ nursery ที่จะพัฒนานั้น จะมีการจัดการคลังสินค้าของทาง nursery และ ระบบ payment สำหรับจัดการกับประวัติการจ่ายเงินค่าเล่าเรียนใน nursery



\section{\ifcpe วัตถุประสงค์ของโครงงาน\else Objectives\fi}
\begin{enumerate}
    \item เพื่อลดความยุ่งยากในการตรวจสอบข้อมูลของเด็กแต่ละคน ซึ่งในการตรวจสอบข้อมูลเด็กแต่ละคนใน แบบเก่า (เอกสารและexcel) กว่าที่เราจะได้ข้อมูลมาต้องผ่านกระบวนการต่างๆ มากมาย อาทิเช่น การนั่งค้นหากองเอกสารเพื่อหาเอกสารที่บันทึกข้อมูลของเด็กที่ต้องการ หรือการมานั่งค้นหาข้อมูลใน excel ซึ่งอาจจะเสียเวลาได้หากมีการบันทึกไฟล์ไว้หลายๆไฟล์ ทาง web application ของเราจะเข้ามาจัดการกับปัญหานี้ โดยทําให้ขั้นตอนนี้เหล่านี้สามารถเข้าถึงข้อมูลได้รวดเร็วและสะดวกยิ่งขึ้น 

    \item สามารถตรวจสอบการเข้าเรียน ตรวจเช็คอุปกรณ์ ตรวจเช็คสุขภาพของเด็กในแต่ละวัน   สามารถเช็คย้อนหลังได้ และ สามารถนําข้อมูลมาแสดงเพื่อใช้ในการตรวจสอบ

    \item เพื่อลดการใช้ทรัพยากรต่างๆลง อาทิ ทรัพยากรกระดาษ ทรัพยากรเวลา และ ทรัพยากรอื่นๆ ที่ไม่จําเป็น
    \item เพื่อทำให้สามารถตรวจประวัติการจ่ายค่าเทอมของผู้ปกครอง แต่ละคนได้แบบออนไลน์ผ่านทางระบบ payment โดยไม่ต้องไปไล่หาเอกสารเหมือนแบบเก่า

\end{enumerate}


\section{\ifcpe ขอบเขตของโครงงาน\else Project scope\fi}




\subsection{\ifcpe ขอบเขตด้านซอฟต์แวร์\else Software scope\fi}
% \begin{itemize}
%     \item 
    เราจะพัฒนาระบบ ทั้งหมด 7 ระบบหลักๆ ได้แก่
    \begin{itemize}
        \item หน้าจัดการประวัติเด็ก คือ หน้าที่ใช้จัดการเกี่ยวกับประวัติของเด็ก สามารถแก้ไขข้อมูลเด็กและทำการย้ายห้องหรือเลื่อนชั้นเรียนเด็กได้
        \item หน้าลงทะเบียนเด็ก คือ หน้ากรอกใบสมัครเข้าเรียนกับ nursery
        \item หน้าเช็คชื่อของเด็ก คือ หน้าเช็คว่าเด็กมาเรียนหรือไม่มาเรียน
        \item หน้าเช็คของเด็ก คือ หน้าเช็คอุปกรณ์เด็ก เช่น ขวดน้ำ ขวดนม ผ้าขนหนู เป็นต้น
        \item หน้าเช็คสุขภาพเด็ก คือ หน้าเช็คสุขภาพเด็ก เช่น วัดไข้ บาดแผล เป็นต้น
        \item หน้าจัดการคลังสินค้า คือ หน้าสำหรับจัดการกับสินค้าหรืออุปกรณ์ใน nursery เช่น ชุดเด็ก กระเป๋า 
        \item หน้าจัดการประวัติการชำระเงินค่าเรียน คือ หน้าสำหรับจัดการประวัติการชำระค่าเทอมของผู้ปกครองเด็ก
    \end{itemize}
% \end{itemize}

\section{\ifcpe ประโยชน์ที่ได้รับ\else Expected outcomes\fi}
จะช่วยลดความล่าช้าในการเรียกใช้ข้อมูลต่างๆลง    
  เนื่องจากไม่ต้องไปค้าหาเอกสารข้อมูลต่างๆ ซึ่งจะช่วยลดการใช้ทรัพยากรของกระดาษลง  ลดขั้นตอนที่ไม่จำเป็นในการย้ายห้องเด็กแต่ละคน ยิ่งไปกว่านั้นการจัดการข้อมูลแบบเก่า  จะต้องมีการกรอกเอกสารและมีการยื่นทำเรื่องเป็นเวลานานพอสมควร แต่ทางตัว web application ของทางเรา จะช่วยลดทอนขั้นตอนที่ไม่จำเป็นต่างๆนั้นออกไป เพื่อให้ทางบุคลากรภายใน nursery สามารถเข้าถึงได้สะดวก

\section{\ifcpe เทคโนโลยีและเครื่องมือที่ใช้\else Technology and tools\fi}

\subsection{\ifcpe เทคโนโลยีด้านฮาร์ดแวร์\else Hardware technology\fi}


\subsection{\ifcpe เทคโนโลยีด้านซอฟต์แวร์\else Software technology\fi}
\begin{itemize}
    \item Frontend : React 	
    \item Backend : Express.js 
    \item Database : MongoDB
    \item Cloud Hosting : Amazon Web Services (AWS)
\end{itemize}



\section{\ifcpe แผนการดำเนินงาน\else Project plan\fi}
%  ถ้าเปิดcommentผมรันตอนนี้แล้วมัน errorอะไรไม่รู้ครับ TT

\begin{plan}{11}{2020}{4}{2021}
    \planitem{11}{2020}{12}{2020}{ออกแบบหน้าเว็บด้วย Adobe XD}
    \planitem{11}{2020}{12}{2020}{ออกแบบ Database Schema}
    \planitem{11}{2020}{12}{2020}{สร้าง Database}
    \planitem{12}{2020}{1}{2021}{เขียน API ที่ฝั่ง Backend}
    \planitem{12}{2020}{1}{2021}{เขียน Frontend แล้ว ทดลองยิง API มาใช้}
    \planitem{12}{2020}{1}{2021}{เชื่อมต่อโค้ดระหว่างฝั่ง Frontend กับ Backend ผ่าน API}
    \planitem{2}{2021}{3}{2021}{ทดสอบระบบ}
    \planitem{2}{2021}{3}{2021}{optimize code}
    \planitem{3}{2021}{4}{2021}{เขียน final report}
\end{plan}

\section{\ifcpe บทบาทและความรับผิดชอบ\else Roles and responsibilities\fi}
ในส่วนนี้ทางผู้พัฒนามีการแบ่งงานออกเป็นสองฝั่ง ก็คือ ฝั่งหน้าบ้าน 
(Frontend)และฝั่งหลังบ้าน (Backend) ซึ่งในฝั่งหน้าบ้านจำเป็นจะต้องมีความรู้ในเรื่องของ HTML, CSS, JS พอสมควร มีความเข้าใจในเรื่องของการออกแบบ UX/UI เพื่อให้ตอบสนองต่อความต้องของลูกค้า และสามารถเขียน requests ส่งมายังฝั่ง Backend ได้ สามารถจัดการกับ response จากฝั่งหลังบ้านได้ ส่วนความรู้ของฝั่งหลังบ้านที่จำเป็นต้องมีความรู้ในเรื่องของการออกแบบฐานข้อมูล  และมีความรู้ในเรื่องการเขียน API  สำหรับใช้จัดการกับ requests ที่ทางฝั่งหน้าบ้านส่งมาเพื่อต้องการที่จะนำไปใช้งาน เช่น ต้องเรียกข้อมูลเพื่อในไปแสดงผล (GET)
ต้องการบันทึกลงบนฐานข้อมูล (POST) ต้องการแก้ไขข้อมูลบนฐานข้อมูล (PUT/PATCH)

งานในฝั่ง Frontend นายพงศ์ภัค เกษมศรี ณ อยุธยา จะเป็นผู้รับผิดชอบ 
ส่วนงานในฝั่ง Backend นายศุภกร หอมนาน  จะเป็นผู้รับผิดชอบ


\section{\ifcpe%
ผลกระทบด้านสังคม สุขภาพ ความปลอดภัย กฎหมาย และวัฒนธรรม
\else%
Impacts of this project on society, health, safety, legal, and cultural issues
\fi}

ในส่วนนี้ในเรื่องผลกระทบที่อาจจะเกิดได้ ก็จะมีในเรื่องของประวัติต่างๆของเด็กในเรื่องนี้เป็นเรื่องที่ละเอียดน้อย ทางผู้พัฒนาและผู้ดูแลระบบอาจถูกมองว่ามีข้อมูลต่างๆของเด็ก แล้วอาจจะนำไปใช้ในทางที่ไม่ดี  ซึ่งจากข้อสันนิฐานเหล่านี้ จะนำไปสู่ผลกระทบในด้านความปลอดภัยและ กฏหมายที่จะเข้ามามีส่วนเกี่ยวข้องในเรื่องนี้ ซึ่งในความเป็นจริงไม่เป็นอย่างที่กล่าวไว้ข้างต้นอย่างแน่นอนที่กล่าวไว้คือ ในกรณีสมมติ 

ในส่วนของผลกระทบด้านบวก คือ การที่ผู้สามารถสามารถทำงานต่างๆได้สะดวกยิ่งขึ้น เช่น การเช็คชื่อ การตรวจสุขภาพ การเช็คอุปกรณ์ สามารถทำผ่าน web application ได้โดยไม่ต้องตรวจเช็คลงในกระดาษ ซึ่งส่งผลให้ลดการใช้กระดาษลง

