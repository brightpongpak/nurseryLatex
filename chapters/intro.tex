\chapter{\ifcpe บทนำ\else Introduction\fi}

\section{\ifcpe ที่มาของโครงงาน\else Project rationale\fi}

ในปัจจุบัน เทคโนโลยีได้เข้ามามีบทบาทในชีวิตประจำวันในทุกๆ ด้าน
ทางด้านการศึกษาก็มีการนำเทคโนโลยีมาใช้ทดแทนแรงงานคนเพื่ออำนวยความสะดวกในการทำงานของเจ้าหน้าที่ฝ่ายต่างๆ
ยกตัวอย่างเช่น เว็บไซต์ลงทะเบียนสอบ GAT/PAT \CI{เว็บไซต์โรงเรียน}{อำนวยความสะดวกอย่างไร}และอื่นๆ
ทางผู้จัดทำจึงได้คิดจะสร้าง\CI{เว็บ}{เราจำเป็นต้องโฟกัสที่เว็บไหม หรือระบบโดยทั่วไปก็พอ ดูเหมือนว่าเราจะให้ความสำคัญกับเว็บมากเกินไปในการประยุกต์ใช้เทคโนโลยีที่กำลังพูดถึง}สำหรับจัดการภายในเนอสเซอรี่
เพราะว่า เว็บไซต์จำพวกสถานศึกษาจะมีเพียงแค่การแสดงข้อมูลสำหรับผู้ใช้ภายนอก
ส่วนการจัดการภายในส่วนมากนั้นยังคงใช้คนในการจดบันทึกข้อมูลลงบนกระดาษอยู่
พวกเราจึงต้องการที่จะทำ\CI{ระบบจัดการภายในนี้ให้อยู่ในรูปแบบเว็บไซต์}{พูดซ้ำกับข้างบน}เพื่อให้เกิดความรวดเร็วในการกรอกข้อมูล การแสดงข้อมูลเมื่อต้องการใช้ข้อมูลนั้นๆ
ซึ่งถ้าเป็นกระดาษการจะนำข้อมูลมาแสดงอาจมีความยุ่งยากหรือล่าช้าในการหาและนำไปแสดงเพื่อยืนยันข้อมูลเหล่านั้น
\CI{ทางเราจึงได้จัดทำระบบเพื่อใช้จัดการภายในเนอสเซอรี่เพื่อมาจัดการกับปัญหาเหล่านี้}{ซ้ำอีกแล้ว}


\section{\ifcpe วัตถุประสงค์ของโครงงาน\else Objectives\fi}
\begin{enumerate}
    \item เพื่อลดความยุ่งยากในการตรวจสอบข้อมูลของเด็กแต่ละคน ซึ่งในการตรวจสอบข้อมูลเด็กแต่ละคนในแบบเก่า (เอกสารและ Excel) กว่าที่เราจะได้ข้อมูลมาต้องผ่านกระบวนการต่างๆมากมาย อาทิเช่น การนั่งค้นหากองเอกสารข้อมูลเพื่อหาเอกสารที่บันทึกข้อมูลของเด็กที่ต้องการ หรือการมานั่งค้นหาข้อมูลใน Excel 
    ซึ่งอาจจะเสียเวลาได้หากมีการบันทึกไฟล์ไว้หลายๆ ไฟล์
    ทาง web application ของเราจะเข้ามาจัดการกับปัญหานี้ โดยทำให้ขั้นตอนนี้เหล่านี้สามารถเข้าถึงข้อมูลได้รวดเร็วและสะดวกยิ่งขึ้น
    \item สามารถตรวจสอบการเข้าเรียน ตรวจเช็คอุปกรณ์ของเด็กได้ในแต่ละวันและนำข้อมูลมาแสดงเพื่อใช้ในการตรวจสอบ
    \item เพื่อลดการใช้ทรัพยากรต่างๆ ลง อาทิ ทรัพยากรกระดาษ ทรัพยากรเวลา และ \CI{ทรัพยากรณ์}{เขียนผิด}อื่นๆ ที่\CI{ไม่จำเป็นลง}{พูดซ้ำ}
\end{enumerate}
\CIreply{พูดถึงเรื่องระบบบัญชี?}

\section{\ifcpe ขอบเขตของโครงงาน\else Project scope\fi}

\subsection{\ifcpe ขอบเขตด้านฮาร์ดแวร์\else Hardware scope\fi}
\begin{itemize}
    \item เราจะพัฒนาให้สามารถใช้ได้ใน screen ขั้นต่ำ 375*667 ขึ้นไป \CIreply{พูดถึงอุปกรณ์ที่รองรับ?}
\end{itemize}

\subsection{\ifcpe ขอบเขตด้านซอฟต์แวร์\else Software scope\fi}
\begin{itemize}
    \item เราจะพัฒนา\CI{ฟิวเจอร์}{features}ทั้งหมด 7 ฟิวเจอร์ได้แก่ \CI{profile register attendance gadget health stock payment}{อธิบายว่าคืออะไรบ้าง ทำเป็น itemize ก็ได้}
\end{itemize}

\section{\ifcpe ประโยชน์ที่ได้รับ\else Expected outcomes\fi}
\begin{itemize}
\item ความสะดวกรวดเร็วในการเรียกใช้ข้อมูลที่จำเป็นของเด็กแต่ละคน
\item ลดขั้นตอนในกระบวนการย้ายห้องของเด็กแต่ละคน ซึ่งแบบเก่าจะมี\CI{ขั้นตอนต่างๆมากมาย}{ยังไม่เห็นภาพ น่าจะต้องเขียนเพิ่มในที่มาของโครงงาน}
 แต่ทางweb appของเราจะลดความยุ่งยากเหล่านั้นลดให้ง่ายและเร็วมากยิ่งขึ้น
 \item ลดการใช้ทรัพยากร(กระดาษ)

\end{itemize}
\CIreply{น่าจะต้องปรับเป็นร้อยแก้วในภายหลัง}

\section{\ifcpe เทคโนโลยีและเครื่องมือที่ใช้\else Technology and tools\fi}

\subsection{\ifcpe เทคโนโลยีด้านฮาร์ดแวร์\else Hardware technology\fi}
\begin{itemize}
    \item Notebook
    \item Ipad
\end{itemize}
\CIreply{อันนี้พูดถึง end-user products แต่เราใช้เครื่องมือที่เป็นฮาร์ดแวร์อะไรบ้าง}

\subsection{\ifcpe เทคโนโลยีด้านซอฟต์แวร์\else Software technology\fi}
\begin{itemize}
    \item Frontend: React, Material UI, Boostrap Versions 4, SweetAlert2		
    \item Backend:Node.js,Express.js
    \item Database : MongoDB
    \item  Cloud Hosting : Firebase(Client),Heruko(Server)
\end{itemize}



\section{\ifcpe แผนการดำเนินงาน\else Project plan\fi}

\begin{plan}{6}{2020}{2}{2021}
    \planitem{6}{2020}{8}{2020}{ศึกษาค้นคว้า}
    \planitem{6}{2020}{6}{2020}{ออกแบบ}
    \planitem{7}{2020}{1}{2021}{พัฒา}
    \planitem{9}{2020}{2}{2021}{ทดสอบ}
\end{plan}
\CIreply{เขียนละเอียดปลีกย่อยได้มากกว่านี้}

\section{\ifcpe บทบาทและความรับผิดชอบ\else Roles and responsibilities\fi}
อธิบายว่าในการทำงาน นศ. มีการกำหนดบทบาทและแบ่งหน้าที่งานอย่างไรในการทำงาน จำเป็นต้องใช้ความรู้ใดในการทำงานบ้าง

\section{\ifcpe%
ผลกระทบด้านสังคม สุขภาพ ความปลอดภัย กฎหมาย และวัฒนธรรม
\else%
Impacts of this project on society, health, safety, legal, and cultural issues
\fi}

แนวทางและโยชน์ในการประยุกต์ใช้งานโครงงานกับงานในด้านอื่นๆ รวมถึงผลกระทบในด้านสังคมและสิ่งแวดล้อมจากการใช้ความรู้ทางวิศวกรรมที่ได้
