\chapter{\ifcpe บทนำ\else Introduction\fi}

\section{\ifcpe ที่มาของโครงงาน\else Project rationale\fi}

ในปัจจุบันเทคโนโลยีได้เข้ามามีบทบาทในชีวิตประจําวันในทุกๆด้าน โดยในแต่ละด้านก็มีการนําเทคโนโลยี มาใช้ทดแทนแรงงานคน  เพื่ออํานวยความสะดวกในการทํางานของเจ้าหน้าที่ฝ่ายต่างๆ  ยกตัวอย่างเช่น แพลตฟอร์มลงทะเบียนสอบ GAT/PATออนไลน์ Application ที่ใช้จัดการกับระบบต่างๆ ภายในบริษัทหรือองค์กรสำนักงานต่างๆ ระบบจัดการภายในโรงพยาบาล ระบบจองคิวรับคิวภายในร้านอาหาร และ อื่นๆ

แต่ว่าในภาคส่วนของทาง Nursery ยังไม่ค่อยพบเห็นแพลตฟอร์มหรือระบบจัดการภายในแบบออนไลน์อยู่เลย
เพราะว่า จากที่สำรวจมาทาง Nursery ยังคงเป็นองค์กรที่มีขนาดเล็ก จึงทำให้ทางผู้พัฒนาหลายๆกลุ่มมองข้ามปัญหาในจุดนี้  ซึ่งปัญหานี้ก็คือ การจัดการของ Nursery ในแบบเดิม ก็คือ การเก็บผ่านเอกสาร  มีปัญหาที่สังเกตได้ชัดเลย ก็คือ ความล่าช้าในกระบวนการจัดการข้อมูล ( ยื่นเรื่องขอประวัติ, ยื่นเรื่องแก้ไขประวัติ ) และ การสูญหายของเอกสาร หรือ เอกสารมีการเสียหาย  

จากเหตุผลที่กล่าวมาข้างต้นทางผู้จัดทําจึงได้นำแนวคิดนี้มาจะสร้างเป็น Web-Application 
สําหรับจัดการภายในเนอสเซอรี่ ในส่วนการจัดการภายในส่วนมาก แล้วนั้นจะยังคงใช้คนในการจดบันทึกข้อมูลลงบนกระดาษ
อยู่ ทางผู้พัฒนาต้องการที่จะทําระบบจัดการภายในนี้ เพื่อให้ครูใน Nursery สามารถตรวจสอบข้อมูลของเด็ก ได้โดยง่าย และ สะดวกรวดเร็วขึ้น และ  เพื่อลดการสูญหายของข้อมูลระหว่างทาง เช่น เกิดการสูญหายของเอกสารข้อมูลส่วนตัวของเด็ก, เอกสารเกี่ยวกับสุขภาพเด็กเกิดสูญเสีย เนื่องจากเปียกน้ำ เป็นต้น และ ต้องการให้มีการแสดงข้อมูลเมื่อมีคนต้องการเรียกใช้ข้อมูลนั้น ณ เวลานั้น   ซึ่งถ้าเป็นกระดาษการจะนําข้อมูลมาแสดงอาจมีความยุ่งยาก หรือ ล่าช้าในการหา และ นําไปแสดงเพื่อยืนยันข้อมูลเหล่านั้น ในส่วนการทำงานพื้นฐานอีกอย่างของ ทางตัวชิ้นงานของเรา 
ก็คือ การจัดการกับกิจวัตรประจำวันต่างๆ ภายใน Nursery ก็คือ การเช็คชื่อ เช็คของใช้ เช็คสุขภาพ
ซึ่งการกระทำต่างๆนี้ ต้องมีการทำในทุกๆวัน พบว่าใน การเก็บข้อมูลแบบเดิมมีปัญหาที่ความล่าช้าใน การหาเอกสาร และ ปัญหาที่เอกสารเสียหายหรือสูญหายไป ทางผู้พัฒนาได้นำเอาสาเหตุเหล่านี้ มาพัฒนาเป็น ระบบเช็ครายวันที่ใช้ง่าย และ สะดวก เพื่อที่จะลดปัญหาเหล่านั้นลด และ ในส่วนของสองระบบสุดท้ายที่ทางเรา คิดจะพัฒนาก็คือ ระบบจัดการคลังสินค้าของทาง Nursery และ ระบบ Payment สำหรับจัดการกับประวัติการจ่ายเงินค่าเล่าเรียนใน Nursery



\section{\ifcpe วัตถุประสงค์ของโครงงาน\else Objectives\fi}
\begin{enumerate}
    \item เพื่อลดความยุ่งยากในการตรวจสอบข้อมูลของเด็กแต่ละคน ซึ่งในการตรวจสอบข้อมูลเด็กแต่ละคนใน แบบเก่า (เอกสารและ Excel) กว่าที่เราจะได้ข้อมูลมาต้องผ่านกระบวนการต่างๆมากมาย อาทิเช่น การนั่งค้นหากองเอกสารข้อมูลเพื่อหาเอกสารที่บันทึกข้อมูลของเด็กที่ต้องการ หรือการมานั่งค้นหาข้อมูลใน Excel ซึ่งอาจจะเสียเวลาได้หากมีการบันทึกไฟล์ไว้หลายๆ ไฟล์ ทาง Web Application ของ เราจะเข้ามาจัดการกับปัญหานี้ โดยทําให้ขั้นตอนนี้เหล่านี้สามารถเข้าถึงข้อมูลได้รวดเร็วและสะดวกยิ่ง ขึ้น 

    \item สามารถตรวจสอบการเข้าเรียน ตรวจเช็คอุปกรณ์ และตรวจเช็คสุขภาพของเด็กได้ในแต่ละวัน และ  สามารถเช็คย้อนหลังได้ด้วย และ สามารถนําข้อมูลมาแสดงเพื่อใช้ในการตรวจสอบ

    \item เพื่อลดการใช้ทรัพยากรต่างๆ ลง อาทิ ทรัพยากรกระดาษ ทรัพยากรเวลา และ ทรัพยากรอื่นๆ ที่ไม่จําเป็นลง
    \item เพื่อทำให้สามารถตรวจประวัติการจ่ายค่าเทอมของผู้ปกครองแต่ละคนได้แบบออนไลน์ผ่านทางระบบ payment โดยไม่ต้องไปไล่หาเอกสารเหมือนแบบเก่า

\end{enumerate}


\section{\ifcpe ขอบเขตของโครงงาน\else Project scope\fi}

\subsection{\ifcpe ขอบเขตด้านฮาร์ดแวร์\else Hardware scope\fi}


\subsection{\ifcpe ขอบเขตด้านซอฟต์แวร์\else Software scope\fi}
% \begin{itemize}
%     \item 
    เราจะพัฒนาระบบ ทั้งหมด 7 ระบบหลักๆ ได้แก่
    \begin{itemize}
        \item หน้าจัดการประวัติเด็ก คือ หน้าที่ใช้จัดการเกี่ยวกับประวัติของเด็ก
        \item หน้าลงทะเบียนเด็ก คือ หน้าสำหรับลงทะเบียนข้อมูลต่างๆของเด็ก
        \item หน้าเช็คชื่อของเด็ก คือ หน้าเช็คชื่อเด็ก
        \item หน้าเช็คของเด็ก คือ หน้าเช็คอุปกรณ์เด็ก
        \item หน้าเช็คสุขภาพเด็ก คือ หน้าเช็คสุขภาพเด็ก
        \item หน้าจัดการคลังสินค้า คือ หน้าสำหรับจัดการกับสินค้าหรืออุปรณ์ใน Nursery เช่น ชุดนักเรียน กระเป๋า 
        \item หน้าจัดการประวัติการชำระเงินค่าเรียน คือ หน้าสำหรับจัดการประวัติการชำระค่าเทอมของผู้ปกครองเด็ก
    \end{itemize}
% \end{itemize}

\section{\ifcpe ประโยชน์ที่ได้รับ\else Expected outcomes\fi}
จะช่วยลดความล่าช้าในการเรียกใช้ข้อมูลต่างๆลง    
  เนื่องจากไม่ต้องไปไล่หาเอกสารข้อมูลต่างๆมากมาย  เหมือนแต่ก่อน ซึ่งจะช่วยลดการใช้ทรัพยากรอย่างกระดาษลง และ 
ลดขั้นตอนที่ไม่จำเป็นในการย้ายห้องเด็ก  แต่ละคน   ซึ่งในการจัดการข้อมูลแบบเก่า  จะต้องมีการกรอกเอกสาร และ มีการยื่นทำเรื่องเป็นเวลานานพอสมควร แต่ทางตัว Web-Application ของทางเรา  จะช่วยลดทอนขั้นตอนที่ไม่จำเป็นต่างๆนั้นออกไป  เพื่อช่วยลดความยุ่งยากลด และ เพื่อให้ทางบุคลากรภายใน Nursery สามารถเข้าถึงได้สะดวก และ ใช้งานง่าย

\section{\ifcpe เทคโนโลยีและเครื่องมือที่ใช้\else Technology and tools\fi}

\subsection{\ifcpe เทคโนโลยีด้านฮาร์ดแวร์\else Hardware technology\fi}


\subsection{\ifcpe เทคโนโลยีด้านซอฟต์แวร์\else Software technology\fi}
\begin{itemize}
    \item Frontend : React 	
    \item Backend : Node.js , Express.js 
    \item Database : MongoDB
    \item Cloud Hosting : Amazon Web Services ( AWS )
\end{itemize}



\section{\ifcpe แผนการดำเนินงาน\else Project plan\fi}

\begin{plan}{6}{2020}{3}{2021}
    \planitem{6}{2020}{8}{2020}{ศึกษาค้นคว้าข้อมูล}
    \planitem{7}{2020}{8}{2020}{สำรวจ / เก็บ requirement จาก Client}
    \planitem{7}{2020}{8}{2020}{วิเคราะห์ requirement ทั้งหมดที่ได้ แล้วคัดเหลือแค่ตัวสำคัญหลักๆ}
    \planitem{9}{2020}{11}{2020}{ออกแบบหน้าเว็บด้วย Adobe XD}
    \planitem{9}{2020}{11}{2020}{ออกแบบ Database Schema}
    \planitem{10}{2020}{12}{2020}{สร้าง Database}

    \planitem{12}{2020}{2}{2021}{เขียน Api ที่ฝั่ง Backend}
    \planitem{12}{2020}{2}{2021}{เขียน Frontend แล้ว ทดลองยิง Api มาใช้}
    \planitem{1}{2021}{3}{2021}{เชื่อมต่อโค้ดระหว่างฝั่ง Frontend กับ Backend ผ่าน Api}

    \planitem{1}{2021}{3}{2021}{ทดสอบระบบ}
    \planitem{8}{2020}{11}{2020}{เขียนรายงาน survey}
    \planitem{12}{2020}{3}{2021}{เขียนรายงาน project}
    \planitem{2}{2021}{3}{2021}{optimize code}


\end{plan}


\section{\ifcpe บทบาทและความรับผิดชอบ\else Roles and responsibilities\fi}
ในส่วนนี้ทางผู้พัฒนามีการแบ่งการดูแลงานเป็น 2 ฝั่ง ก็คือ ฝั่งหน้าบ้าน 
( Frontend ) และ ฝั่งหลังบ้าน ( Backend ) ซึ่งในฝั่งหน้าบ้านจำเป็นจะต้องมีความรู้ในเรื่องของ HTML,CSS,JS พอสมควร มีความเข้าใจในเรื่องของการออกแบบ UX/UI เพื่อให้ตอบสนองต่อความต้องของลูกค้า และสามารถเขียน Requests ส่งมายังฝั่ง Backendได้ สามารถจัดการกับ Response จากฝั่งหลังบ้านได้ ส่วนความรู้ของฝั่งหลังบ้านที่จำเป็นต้องมีความรู้ในเรื่องของการออกแบบฐานข้อมูลเพื่อให้รองรับกับ หน้า UX/UI ที่ฝั่ง Frontendเป็นคนออกแบบ และมีความรู้ในเรื่องการเขียน api สำหรับใช้จัดการกับ Requests ที่ทางฝั่งหน้าบ้านส่งมาเพื่อต้องการจะทำอะไรต่อมิอะไร เช่น ต้องเรียกข้อมูลเพื่อในไปแสดงผม ( GET )
ต้องการบันทึกลงบนฐานข้อมูล (POST ) ต้องการแก้ไขข้อมูลบนฐานข้อมูล ( PUT )
งานในฝั่ง Frontend นายพงศ์ภัค เกษมศรี ณ อยุธยา จะเป็นผู้รับผิดชอบ 
ส่วนงานในฝั่ง Backend นายศุภกร หอมนาน  จะเป็นผู้รับผิดชอบ


\section{\ifcpe%
ผลกระทบด้านสังคม สุขภาพ ความปลอดภัย กฎหมาย และวัฒนธรรม
\else%
Impacts of this project on society, health, safety, legal, and cultural issues
\fi}

ในส่วนนี้ในเรื่องผลกระทบที่อาจจะเกิดได้ก็จะมีในเรื่องของประวัติต่างๆของเด็กในเรื่องนี้เป็นเรื่องที่ละเอียดน้อย ทางผู้พัฒนา และ ผู้ดูแลระบบอาจถูกมองว่ามีข้อมูลต่างๆ ของเด็กอาจจะนำไปใช้ในทางที่ไม่ดีได้  ซึ่งจากข้อสันนิฐานเหล่านี้ จะนำไปสู่ผลกระทบในด้านความปลอดภัย และ กฏหมายที่จะเข้ามามีส่วนเกี่ยวข้องในเรื่องนี้ ซึ่งในความเป็นจริงไม่เป็นอย่างที่กล่าวไว้ข้างต้นอย่างแน่นอนที่กล่าวไว้คือ ในกรณีสมมติ 

ในส่วนของผลกระทบด้านบวก คือ การที่ผู้สามารถสามารถทำงานต่างๆ ได้สะดวกยิ่งขึ้น เช่น การเช็คชื่อ การตรวจสุขภาพ การเช็คอุปกรณ์ สามารถทำผ่าน web-Application ได้โดยไม่ต้องตรวจเช็คลงในกระดาษอีกต่อไป ซึ่งส่งผลให้ลดการใช้กระดาษลงอีกด้วย

