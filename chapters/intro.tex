\chapter{\ifcpe บทนำ\else Introduction\fi}

\section{\ifcpe ที่มาของโครงงาน\else Project rationale\fi}

ในปัจจุบันเทคโนโลยีได้เข้ามามีบทบาทในชีวิตประจําวันในทุกๆ ด้าน โดยในแต่ละด้านก็มีการนําเทคโนโลยีมาใช้ทดแทนแรงงานคน  เพื่ออํานวยความสะดวกในการทํางานของเจ้าหน้าที่ฝ่ายต่างๆ  ยกตัวอย่างเช่น แพลตฟอร์มลงทะเบียนสอบ GAT/PAT ออนไลน์ application ที่ใช้จัดการกับระบบต่างๆ ภายในบริษัทหรือองค์กร สำนักงานต่างๆ ระบบจัดการภายในโรงพยาบาล ระบบจองคิวรับคิวภายในร้านอาหาร เป็นต้น

แต่ว่าในภาคส่วนของทาง nursery ยังไม่ค่อยพบเห็นแพลตฟอร์มหรือระบบจัดการภายในแบบออนไลน์เท่าใดนัก
เนื่องจาก nursery เป็นองค์กรที่มีขนาดเล็ก ไม่ได้มีระบบการบริหารจัดการจากส่วนกลาง จึงทำให้ผู้พัฒนาหลายๆ กลุ่มมองไม่เห็นความสำคัญของปัญหาที่เกี่ยวกับการจัดการ nursery ได้แก่ การจัดเก็บข้อมูลต่างๆ รายวัน ซึ่งในปัจจุบันยังทำลงบนกระดาษอยู่ ส่งผลให้เกิดปัญหาความล่าช้าในกระบวนการรวบรวมและจัดเก็บข้อมูล
นอกจากนี้ การแก้ไขหรือเข้าถึงข้อมูลต่างๆ เช่น การตรวจสอบและแก้ไขประวัตินักเรียน ย่อมเกิดความล่าช้า เนื่องจากจำเป็นต้องดำเนินการผ่านเจ้าหน้าที่ของ nursery ยิ่งไปกว่านั้น ยังมีเรื่องของความเสี่ยงในการที่เอกสารที่จัดเก็บไว้จะสูญหายหรือสูญเสียอีกด้วย

จากเหตุผลที่กล่าวมาข้างต้น ทางผู้จัดทําจึงได้นำแนวคิดนี้มาสร้างเป็น web application 
สําหรับจัดการภายในเนอสเซอรี่ เพื่อให้ครูและเจ้าหน้าที่ใน nursery สามารถตรวจสอบและแสดงข้อมูลของเด็กได้โดยง่ายและสะดวกรวดเร็วขึ้น และเพื่อลดการสูญหายของข้อมูลระหว่างการจัดส่งหรือจัดเก็บ ที่อาจเกิดขึ้นได้จากการใช้กระดาษ
นอกเหนือจากการจัดเก็บข้อมูลส่วนบุคคลของเด็กแต่ละคนแล้ว ระบบจัดการภายใน nursery ที่ผู้จัดทำได้วางแผนพัฒนานั้น จะสามารถรองรับการจัดการกิจวัตรประจำวันต่างๆ ภายใน nursery ได้แก่ การเช็คชื่อ เช็คของใช้ เช็คสุขภาพ
ซึ่งสามารถทำได้แบบ real-time ผ่านระบบออนไลน์
ในท้ายที่สุด ระบบจัดการ nursery ที่จะพัฒนานั้น จะมีการจัดการคลังสินค้าของทาง Nursery และ ระบบ Payment สำหรับจัดการกับประวัติการจ่ายเงินค่าเล่าเรียนใน Nursery



\section{\ifcpe วัตถุประสงค์ของโครงงาน\else Objectives\fi}
\begin{enumerate}
    \item เพื่อลดความยุ่งยากในการตรวจสอบข้อมูลของเด็กแต่ละคน ซึ่งในการตรวจสอบข้อมูลเด็กแต่ละคนใน แบบเก่า (เอกสารและ Excel) กว่าที่เราจะได้ข้อมูลมาต้องผ่านกระบวนการต่างๆ มากมาย อาทิเช่น การนั่งค้นหากองเอกสารข้อมูลเพื่อหาเอกสารที่บันทึกข้อมูลของเด็กที่ต้องการ หรือการมานั่งค้นหาข้อมูลใน Excel ซึ่งอาจจะเสียเวลาได้หากมีการบันทึกไฟล์ไว้หลายๆ ไฟล์ ทาง Web Application ของเราจะเข้ามาจัดการกับปัญหานี้ โดยทําให้ขั้นตอนนี้เหล่านี้สามารถเข้าถึงข้อมูลได้รวดเร็วและสะดวกยิ่งขึ้น 

    \item สามารถตรวจสอบการเข้าเรียน ตรวจเช็คอุปกรณ์ และตรวจเช็คสุขภาพของเด็กได้ในแต่ละวัน และ  สามารถเช็คย้อนหลังได้ด้วย และ สามารถนําข้อมูลมาแสดงเพื่อใช้ในการตรวจสอบ

    \item เพื่อลดการใช้ทรัพยากรต่างๆ ลง อาทิ ทรัพยากรกระดาษ ทรัพยากรเวลา และ ทรัพยากรอื่นๆ ที่ไม่จําเป็นลง
    \item เพื่อทำให้สามารถตรวจประวัติการจ่ายค่าเทอมของผู้ปกครองแต่ละคนได้แบบออนไลน์ผ่านทางระบบ payment โดยไม่ต้องไปไล่หาเอกสารเหมือนแบบเก่า

\end{enumerate}


\section{\ifcpe ขอบเขตของโครงงาน\else Project scope\fi}

\subsection{\ifcpe ขอบเขตด้านฮาร์ดแวร์\else Hardware scope\fi}


\subsection{\ifcpe ขอบเขตด้านซอฟต์แวร์\else Software scope\fi}
% \begin{itemize}
%     \item 
    เราจะพัฒนาระบบ ทั้งหมด 7 ระบบหลักๆ ได้แก่
    \CIreply{fix}
    \begin{itemize}
        \item หน้าจัดการประวัติเด็ก คือ หน้าที่ใช้จัดการเกี่ยวกับประวัติของเด็ก\CIreply{ย้ายห้อง}
        \item หน้าลงทะเบียนเด็ก คือ หน้าสำหรับลงทะเบียนข้อมูลต่างๆของเด็ก
        \item หน้าเช็คชื่อของเด็ก คือ หน้าเช็คชื่อเด็ก
        \item หน้าเช็คของเด็ก คือ หน้าเช็คอุปกรณ์เด็ก
        \item หน้าเช็คสุขภาพเด็ก คือ หน้าเช็คสุขภาพเด็ก
        \item หน้าจัดการคลังสินค้า คือ หน้าสำหรับจัดการกับสินค้าหรืออุปรณ์ใน Nursery เช่น ชุดนักเรียน กระเป๋า 
        \item หน้าจัดการประวัติการชำระเงินค่าเรียน คือ หน้าสำหรับจัดการประวัติการชำระค่าเทอมของผู้ปกครองเด็ก
    \end{itemize}
% \end{itemize}

\section{\ifcpe ประโยชน์ที่ได้รับ\else Expected outcomes\fi}
จะช่วยลดความล่าช้าในการเรียกใช้ข้อมูลต่างๆ ลง    
  เนื่องจากไม่ต้องไปไล่หาเอกสารข้อมูลต่างๆ มากมาย  เหมือนแต่ก่อน ซึ่งจะช่วยลดการใช้ทรัพยากรอย่างกระดาษลง และ ลดขั้นตอนที่ไม่จำเป็น\CI{ในการย้ายห้องเด็ก}{มายังไง}  แต่ละคน   ซึ่งในการจัดการข้อมูลแบบเก่า  จะต้องมีการกรอกเอกสาร และ มีการยื่นทำเรื่องเป็นเวลานานพอสมควร แต่ทางตัว Web-Application ของทางเรา  จะช่วยลดทอนขั้นตอนที่ไม่จำเป็นต่างๆนั้นออกไป  เพื่อช่วยลดความยุ่งยากลด และ เพื่อให้ทางบุคลากรภายใน Nursery สามารถเข้าถึงได้สะดวก และ ใช้งานง่าย

\section{\ifcpe เทคโนโลยีและเครื่องมือที่ใช้\else Technology and tools\fi}

\subsection{\ifcpe เทคโนโลยีด้านฮาร์ดแวร์\else Hardware technology\fi}


\subsection{\ifcpe เทคโนโลยีด้านซอฟต์แวร์\else Software technology\fi}
\begin{itemize}
    \item Frontend : React 	
    \item Backend : Node.js , Express.js 
    \item Database : MongoDB
    \item Cloud Hosting : Amazon Web Services (AWS)
\end{itemize}



\section{\ifcpe แผนการดำเนินงาน\else Project plan\fi}

\begin{plan}{6}{2020}{3}{2021}
    \planitem{6}{2020}{8}{2020}{ศึกษาค้นคว้าข้อมูล}
    \planitem{7}{2020}{8}{2020}{สำรวจ / เก็บ requirement จาก Client}
    \planitem{7}{2020}{8}{2020}{วิเคราะห์ requirement ทั้งหมดที่ได้ แล้วคัดเหลือแค่ตัวสำคัญหลักๆ}
    \planitem{9}{2020}{11}{2020}{ออกแบบหน้าเว็บด้วย Adobe XD}
    \planitem{9}{2020}{11}{2020}{ออกแบบ Database Schema}
    \planitem{10}{2020}{12}{2020}{สร้าง Database}

    \planitem{12}{2020}{2}{2021}{เขียน Api ที่ฝั่ง Backend}
    \planitem{12}{2020}{2}{2021}{เขียน Frontend แล้ว ทดลองยิง Api มาใช้}
    \planitem{1}{2021}{3}{2021}{เชื่อมต่อโค้ดระหว่างฝั่ง Frontend กับ Backend ผ่าน Api}

    \planitem{1}{2021}{3}{2021}{ทดสอบระบบ}
    \planitem{8}{2020}{11}{2020}{เขียนรายงาน survey}
    \planitem{12}{2020}{3}{2021}{เขียนรายงาน project}
    \planitem{2}{2021}{3}{2021}{optimize code}


\end{plan}


\section{\ifcpe บทบาทและความรับผิดชอบ\else Roles and responsibilities\fi}
ในส่วนนี้ทางผู้พัฒนามีการแบ่งการดูแลงานเป็น 2 ฝั่ง ก็คือ ฝั่งหน้าบ้าน 
( Frontend ) และ ฝั่งหลังบ้าน ( Backend ) ซึ่งในฝั่งหน้าบ้านจำเป็นจะต้องมีความรู้ในเรื่องของ HTML, CSS, JS พอสมควร มีความเข้าใจในเรื่องของการออกแบบ UX/UI เพื่อให้ตอบสนองต่อความต้องของลูกค้า และสามารถเขียน Requests ส่งมายังฝั่ง Backend ได้ สามารถจัดการกับ Response จากฝั่งหลังบ้านได้ ส่วนความรู้ของฝั่งหลังบ้านที่จำเป็นต้องมีความรู้ในเรื่องของการออกแบบฐานข้อมูล\CI{เพื่อให้รองรับกับ หน้า UX/UI}{จริงเหรอ} ที่ฝั่ง Frontendเป็นคนออกแบบ และมีความรู้ในเรื่องการเขียน \CI{api}{check spelling} สำหรับใช้จัดการกับ Requests ที่ทางฝั่งหน้าบ้านส่งมาเพื่อต้องการจะทำอะไรต่อมิอะไร เช่น ต้องเรียกข้อมูลเพื่อในไปแสดงผม ( GET )
ต้องการบันทึกลงบนฐานข้อมูล (POST ) ต้องการแก้ไขข้อมูลบนฐานข้อมูล ( PUT )

งานในฝั่ง Frontend นายพงศ์ภัค เกษมศรี ณ อยุธยา จะเป็นผู้รับผิดชอบ 
ส่วนงานในฝั่ง Backend นายศุภกร หอมนาน  จะเป็นผู้รับผิดชอบ


\section{\ifcpe%
ผลกระทบด้านสังคม สุขภาพ ความปลอดภัย กฎหมาย และวัฒนธรรม
\else%
Impacts of this project on society, health, safety, legal, and cultural issues
\fi}

ในส่วนนี้ในเรื่องผลกระทบที่อาจจะเกิดได้ก็จะมีในเรื่องของประวัติต่างๆของเด็กในเรื่องนี้เป็นเรื่องที่ละเอียดน้อย ทางผู้พัฒนา และ ผู้ดูแลระบบอาจถูกมองว่ามีข้อมูลต่างๆ ของเด็กอาจจะนำไปใช้ในทางที่ไม่ดีได้  ซึ่งจากข้อสันนิฐานเหล่านี้ จะนำไปสู่ผลกระทบในด้านความปลอดภัย และ กฏหมายที่จะเข้ามามีส่วนเกี่ยวข้องในเรื่องนี้ ซึ่งในความเป็นจริงไม่เป็นอย่างที่กล่าวไว้ข้างต้นอย่างแน่นอนที่กล่าวไว้คือ ในกรณีสมมติ 

ในส่วนของผลกระทบด้านบวก คือ การที่ผู้สามารถสามารถทำงานต่างๆ ได้สะดวกยิ่งขึ้น เช่น การเช็คชื่อ การตรวจสุขภาพ การเช็คอุปกรณ์ สามารถทำผ่าน web-Application ได้โดยไม่ต้องตรวจเช็คลงในกระดาษอีกต่อไป ซึ่งส่งผลให้ลดการใช้กระดาษลงอีกด้วย

