\chapter{\ifcpe บทนำ\else Introduction\fi}

\section{\ifcpe ที่มาของโครงงาน\else Project rationale\fi}

ในปัจจุบันเทคโนโลยีได้เข้ามามีบทบาทในชีวิตประจําวันในทุกๆด้าน ทางด้านการศึกษาก็มีการนําเทคโนโลยี มาใช้ทดแทนแรงงานคนเพื่ออํานวยความสะดวกในการทํางานของเจ้าหน้าที่ฝ่ายต่างๆ ยกตัวอย่างเช่น แพลตฟอร์มลงทะเบียนสอบ GAT/PATออนไลน์ application ที่ใช้จัดการกับระบบต่างๆในโรงเรียนและอื่นๆ
\CIreply{ทำไมถึงยกตัวอย่างจากโรงเรียน?์  เดาว่า nursery ก็เป็นการศึกษาแบบหนึ่ง แต่จุดนี้ยังไม่ชัดเจน \enskip อาจจะยกตัวอย่างเชิงสาธารณสุขด้วย หรือด้านอื่นๆ ที่เกี่ยวข้องกับการดำเนินการของ nursery}
แต่ว่าในภาคส่วนของทาง nursery ยังไม่ค่อยพบเห็นแพลตฟอร์มหรือระบบจัดการภายในแบบออนไลน์อยู่เลย
\CIreply{เพราะอะไร}
จากเหตุผลที่กล่าวมาข้างต้นทางผู้จัดทําจึงได้นำแนวคิดนี้มาจะสร้างเป็น web-application 
สําหรับจัดการภายในเนอสเซอรี่ ในส่วนการจัดการภายในส่วนมากแล้วนั้นจะยังคงใช้คนในการจดบันทึกข้อมูลลงบนกระดาษ
อยู่ ทางผู้พัฒนาต้องการที่จะทําระบบจัดการภายในนี้เพื่อลดการสูญหายของข้อมูลระหว่างทาง เช่น เกิดการสูญหายของเอกสารข้อมูลส่วนตัวของเด็ก, เอกสารเกี่ยวกับสุขภาพเด็กเกิดสูญเสียเนื่องจากเปียกน้ำ เป็นต้น และ ต้องการให้มีการแสดงข้อมูลเมื่อมีคนต้องการเรียกใช้ข้อมูลนั้น ณ เวลานั้น   ซึ่งถ้าเป็นกระดาษการจะนําข้อมูลมาแสดงอาจมีความยุ่งยากหรือล่าช้าในการหาและนําไปแสดงเพื่อยืนยันข้อมูลเหล่านั้น 



\section{\ifcpe วัตถุประสงค์ของโครงงาน\else Objectives\fi}
\begin{enumerate}
    \item เพื่อลดความยุ่งยากในการตรวจสอบข้อมูลของเด็กแต่ละคน ซึ่งในการตรวจสอบข้อมูลเด็กแต่ละคนใน แบบเก่า (เอกสารและ Excel) กว่าที่เราจะได้ข้อมูลมาต้องผ่านกระบวนการต่างๆมากมาย\CIreply{เกริ่นให้เห็นภาพก่อนใน 1.1} อาทิเช่น การนั่งค้นหากองเอกสารข้อมูลเพื่อหาเอกสารที่บันทึกข้อมูลของเด็กที่ต้องการ หรือการมานั่งค้นหาข้อมูลใน Excel ซึ่งอาจจะเสียเวลาได้หากมีการบันทึกไฟล์ไว้หลายๆ ไฟล์ ทาง web application ของ เราจะเข้ามาจัดการกับปัญหานี้ โดยทําให้ขั้นตอนนี้เหล่านี้สามารถเข้าถึงข้อมูลได้รวดเร็วและสะดวกยิ่ง ขึ้น 

    \item สามารถตรวจสอบการเข้าเรียน ตรวจเช็คอุปกรณ์ และตรวจเช็คสุขภาพของเด็กได้ในแต่ละวันและนําข้อมูลมาแสดงเพื่อใช้ในการตรวจสอบ\CIreply{ยังไม่ได้อธิบาย workflow ของ nursery daily routine \enskip ควรพูดก่อนใน 1.1}

    \item เพื่อลดการใช้ทรัพยากรต่างๆ ลง อาทิ ทรัพยากรกระดาษ ทรัพยากรเวลา และ ทรัพยากรอื่นๆ ที่ไม่จําเป็นลง
    \item เพื่อทำให้สามารถตรวจประวัติการจ่ายค่าเทอมของผู้ปกครองแต่ละคนได้แบบออนไลน์ผ่านทางระบบ payment โดยไม่ต้องไปไล่หาเอกสารเหมือนแบบเก่า\CIreply{ยังไม่เห็นภาพว่าเกี่ยวข้องกับระบบอย่างไร เขียนให้เห็นภาพก่อนใน 1.1}

\end{enumerate}


\section{\ifcpe ขอบเขตของโครงงาน\else Project scope\fi}

\subsection{\ifcpe ขอบเขตด้านฮาร์ดแวร์\else Hardware scope\fi}
\begin{itemize}
    \item รองรับการใช้งานเฉพาะบนคอมพิวเตอร์หรือโน๊ตบุ๊คเท่านั้น

\end{itemize}

\subsection{\ifcpe ขอบเขตด้านซอฟต์แวร์\else Software scope\fi}
% \begin{itemize}
%     \item 
    เราจะพัฒนา feature ทั้งหมด 7 features ได้แก่\CIreply{ต้องมีภาษาอังกฤษกำกับใน list นี้หรือไม่}
    \begin{itemize}
        \item profile คือ หน้าที่ใช้จัดการเกี่ยวกับประวัติของเด็ก
        \item  register คือ หน้าสำหรับลงทะเบียนข้อมูลต่างๆของเด็ก
        \item attendance คือ หน้าเช็คชื่อเด็ก
        \item gadget คือ หน้าเช็คอุปกรณ์เด็ก
        \item health คือ หน้าเช็คสุขภาพเด็ก
        \item stock คือ หน้าสำหรับจัดการกับ stock ของใน nursery\CIreply{ไม่เข้าใจว่าแปลว่าอะไร}
        \item payment คือ หน้าสำหรับจัดการประวัติการชำระค่าเทอมของผู้ปกครองเด็ก
    \end{itemize}
% \end{itemize}

\section{\ifcpe ประโยชน์ที่ได้รับ\else Expected outcomes\fi}
จะช่วยลดความล่าช้าในการเรียกใช้ข้อมูลต่างๆลง เนื่องจากไม่ต้องไปไล่หาเอกสารข้อมูลต่างๆมากมายเหมือนแต่ก่อน ซึ่งจะช่วยลดการใช้ทรัพยากรอย่างกระดาษลง และ 
ลดขั้นตอนที่ไม่จำเป็นในการย้ายห้องเด็กแต่ละคน ซึ่งในการจัดการข้อมูลแบบเก่าจะต้องมีการกรอกเอกสารและมีการยื่นทำเรื่องเป็นเวลานานพอสมควร แต่ทางตัว web-application ของทางเราจะช่วยลดทอนขั้นตอนที่ไม่จำเป็นต่างๆนั้นออกไปเพื่อช่วยลดความยุ่งยากลด และ เพื่อให้ทางบุคลากรภายใน nursery สามารถเข้าถึงได้สะดวก และ ใช้งานง่าย

\section{\ifcpe เทคโนโลยีและเครื่องมือที่ใช้\else Technology and tools\fi}

\subsection{\ifcpe เทคโนโลยีด้านฮาร์ดแวร์\else Hardware technology\fi}
\begin{itemize}
    \item Notebook\CIreply{ใข้สำหรับส่วนไหน}
\end{itemize}

\subsection{\ifcpe เทคโนโลยีด้านซอฟต์แวร์\else Software technology\fi}
\begin{itemize}
    \item Frontend : React 	
    \item Backend : Node.js , Express.js 
    \item Database : MongoDB
    \item Cloud Hosting : Amazon Web Services ( AWS )
\end{itemize}



\section{\ifcpe แผนการดำเนินงาน\else Project plan\fi}

\begin{plan}{6}{2020}{4}{2021}
    \planitem{6}{2020}{10}{2020}{ศึกษาค้นคว้าข้อมูล}
    \planitem{6}{2020}{7}{2020}{ออกแบบระบบ}
    \planitem{8}{2020}{2}{2021}{พัฒาระบบตามที่ออกแบบไว้}
    \planitem{10}{2020}{4}{2021}{ทดสอบระบบ}
    \planitem{6}{2020}{10}{2020}{เขียนรายงาน survey}
    \planitem{11}{2020}{4}{2021}{เขียนรายงาน project}
    \planitem{2}{2021}{4}{2021}{เก็บรายละเอียดของตัวชิ้นงาน}

\end{plan}
\CIreply{เก็บรายละเอียด แปลว่าอะไร}

\section{\ifcpe บทบาทและความรับผิดชอบ\else Roles and responsibilities\fi}
ในส่วนนี้ทางผู้พัฒนามีการแบ่งการดูแลงานเป็น 2 ฝั่ง ก็คือ ฝั่งหน้าบ้าน 
( Frontend ) และ ฝั่งหลังบ้าน ( Backend ) ซึ่งในฝั่งหน้าบ้านจำเป็นจะต้องมีความรู้ในเรื่องของ HTML,CSS,JS พอสมควร มีความเข้าใจในเรื่องของการออกแบบ UX/UI เพื่อให้ตอบสนองต่อความต้องของลูกค้า และสามารถเขียน Requests ส่งมายังฝั่ง Backendได้ สามารถจัดการกับ Response จากฝั่งหลังบ้านได้ ส่วนความรู้ของฝั่งหลังบ้านที่จำเป็นต้องมีความรู้ในเรื่องของการออกแบบฐานข้อมูลเพื่อให้รองรับกับ หน้า UX/UI ที่ฝั่ง Frontendเป็นคนออกแบบ และมีความรู้ในเรื่องการเขียน api สำหรับใช้จัดการกับ Requests ที่ทางฝั่งหน้าบ้านส่งมาเพื่อต้องการจะทำอะไรต่อมิอะไร เช่น ต้องเรียกข้อมูลเพื่อในไปแสดงผม ( GET )
ต้องการบันทึกลงบนฐานข้อมูล (POST ) ต้องการแก้ไขข้อมูลบนฐานข้อมูล ( PUT )
งานในฝั่ง Frontend นายพงศ์ภัค เกษมศรี ณ อยุธยา จะเป็นผู้รับผิดชอบ 
ส่วนงานในฝั่ง Backend นายศุภกร หอมนาน  จะเป็นผู้รับผิดชอบ


\section{\ifcpe%
ผลกระทบด้านสังคม สุขภาพ ความปลอดภัย กฎหมาย และวัฒนธรรม
\else%
Impacts of this project on society, health, safety, legal, and cultural issues
\fi}

ในส่วนนี้ในเรื่องผลกระทบที่อาจจะเกิดได้ก็จะมีในเรื่องของประวัติต่างๆของเด็กในเรื่องนี้เป็นเรื่องที่ละเอียดน้อย ทางผู้พัฒนาและผู้ดูแลระบบอาจถูกมองว่ามีข้อมูลต่างๆของเด็กอาจจะนำไปใช้ในทางที่ไม่ดีได้ซึ่งจากข้อสันนิฐานเหล่านี้จะนำไปสู่ผลกระทบในด้านความปลอดภัย และ กฏหมายที่จะเข้ามามีส่วนเกี่ยวข้องในเรื่องนี้ ซึ่งในความเป็นจริงไม่เป็นอย่างที่กล่าวไว้ข้างต้นอย่างแน่นอนที่กล่าวไว้คือในกรณีสมมติ
\CIreply{ผลกระทบ แปลว่า impact จึงควรพูดถึงด้านดีด้วย}
