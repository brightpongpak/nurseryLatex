\maketitle
\makesignature

\ifproject
\begin{abstractTH}
เนื่องในปัจุบันการจัดการข้อมูลภายใน Nursery ยังคงใช้วิธีการเก็บข้อมูลแบบเดิมๆ อยู่ซึ่งก็คือการจดลงในกระดาษก่อน\CI{แล้วค่อยนำมากรอกลงในคอมพิวเตอร์}{ทำไมต้องกรอกลงคอมพิวเตอร์} ตัวอย่างของข้อมูล เช่น การเช็คชื่อนักเรียน ตรวจเช็คอุปกร์ของนักเรียน การกรอกใบสมัคร การตรวจสุขภาพ 
\enskip
ซึ่งการเก็บข้อมูลนั้นเก็บไปเพื่อใช้ในการตรวจสอบข้อมูลเด็กแต่ละคนและเก็บข้อมูลสำคัญ(เบอร์ติดต่อผู้ปกครอง โรงพยาบาลฉุกเฉิน ข้อมูลทางการแพทย์ของเด็กแต่ละคน)
\enskip
ซึ่งในการเก็บข้อมูลข้างต้นที่กล่าวมานั้นต้องเก็บโดยเขียนลงในกระดาษก่อนและค่อยนำไปกรอกใส่ในคอมพิวเตอร์ในภายหลังซึ่งทำให้เสียเวลาในการกรอกและสิ้นเปลืองทรัพยากรณ์กระดาษไปโดยใช่เหตุ 
\enskip 
จากปัญหาที่เกิดขึ้นทางผู้พัฒนาจึงคิดระบบจัดการภายในของ Nursery ขึ้นมาเพื่อลดความยุ่งยากในการเก็บข้อมูลลง ซึ่งระบบของเรานั้นการเก็บข้อมูลทำได้โดยการกรอกลงไปในคอมพิวเตอร์โดยผ่านโปรแกรมประยุกต์บนเว็บ(Web Application) ได้โดยตรง
\enskip
นอกจากการแก้ปัญหาข้างต้นแล้วยังมีประโยชน์ในเรื่องของการดูข้อมูลต่างๆ ได้จากภาย นอกที่ทำงาน กล่าวคือเราสามารถดูข้อมูลโดยผ่านเว็บไซต์ของเราได้ต่างจากการเก็บข้อมูลแบบเดิมที่สามารถดูได้เฉพาะที่ทำงานเท่านั้น จากเหตุผลข้างต้นที่กล่าวมาระบบของเราช่วยอำนวยความสะดวกต่อผู้ใช้งาน

\textcolor{Green4}{
การบริหารจัดการเนอสเซอรี่จำเป็นต้องจัดเก็บข้อมูลเพื่อใช้ในการตรวจสอบความพร้อมของเด็กในแต่ละวัน อาทิ สถิติการมา--ขาด อุปกรณ์ส่วนตัว และข้อมูลสุขภาพ รวมไปถึงข้อมูลยามฉุกเฉิน อาทิ หมายเลขติดต่อผู้ปกครอง โรงพยาบาลฉุกเฉิน และข้อมูลส่วนบุคคลเชิงการแพทย์
\enskip
ในปัจจุบัน การเก็บข้อมูลดังกล่าวยังคงใช้วิธีกรอกลงบนกระดาษ ซึ่งเกิดความผิดพลาดได้ง่าย ใช้เวลานาน และสิ้นเปลืองทรัพยากรกระดาษโดยใช่เหตุ
(ถ้ารู้ว่าต้องกรอกลงคอมทำไม เขียนต่อตรงนี้: นอกจากนี้ หลังจากเก็บข้อมูลลงกระดาษแล้ว ทางเนอสเซอรี่เองยังจำเป็นต้องกรอกข้อมูลลงในคอมพิวเตอร์อีกครั้งหนึ่ง เนื่องจาก...)
\enskip
โครงงานนี้มุ่งที่จะแก้ปัญหาดังกล่าว โดยการสร้างโปรแกรมประยุกต์บนเว็บ (web application) สำหรับจัดการเนอสเซอรี่ เพื่อเก็บข้อมูลโดยตรง ทำให้ลดความยุ่งยากในการจัดเก็บข้อมูล ลดความผิดพลาดในการกรอกข้อมูล และยังสามารถนำข้อมูลที่มีอยู่ไปวิเคราะห์และประเมินพัฒนาการของเด็กได้อย่างรวดเร็วอีกด้วย
\enskip
นอกจากนี้ การใช้ระบบจัดการเนอสเซอรี่ยังช่วยอำนวยความสะดวกในการตรวจสอบข้อมูลจากนอกสถานที่ (ทำไมต้องดูนอกเนอสเซอรี่ด้วย)
\enskip
ระบบจัดการเนอสเซอรี่นี้จะช่วยให้การบริหารจัดการเป็นไปอย่างคล่องตัวและเบ็ดเสร็จ ตั้งแต่กระบวนการรับเด็กแรกเข้า การเลื่อนชั้น จนกระทั่งถึงเวลาที่เด็กพร้อมจะเข้าเรียนในระดับอนุบาล
}

data visualization?
\end{abstractTH}

\begin{abstract}
The abstract would be placed here. It usually does not exceed 350 words
long (not counting the heading), and must not take up more than one (1) page
(even if fewer than 350 words long).

Make sure your abstract sits inside the \texttt{abstract} environment.
\end{abstract}

\iffalse
\begin{dedication}
This document is dedicated to all Chiang Mai University students.

Dedication page is optional.
\end{dedication}
\fi % \iffalse

\begin{acknowledgments}
Your acknowledgments go here. Make sure it sits inside the
\texttt{acknowledgment} environment.

\acksign{2020}{5}{25}
\end{acknowledgments}%
\fi % \ifproject

\contentspage

\ifproject
\figurelistpage

\tablelistpage
\fi % \ifproject

% \abbrlist % this page is optional

% \symlist % this page is optional

% \preface % this section is optional
