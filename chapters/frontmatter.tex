\maketitle
\makesignature

\ifproject
\begin{abstractTH}
    การบริหารจัดการเนอสเซอรี่จําเป็นต้องจัดเก็บข้อมูลเพื่อใช้ในการตรวจสอบความพร้อมของเด็กในแต่ละ
    วัน อาทิ สถิติการมา–ขาด อุปกรณ์ส่วนตัว และข้อมูลสุขภาพ รวมไปถึงข้อมูลยามฉุกเฉิน อาทิ หมายเลขติดต่อ ผู้ปกครอง โรงพยาบาลฉุกเฉิน และข้อมูลส่วนบุคคล ในปัจจุบัน การเก็บข้อมูลดังกล่าวยังคง ใช้วิธีกรอกลงบนเอกสาร ซึ่งเกิดความผิดพลาดได้ง่าย ใช้เวลานาน และสิ้นเปลืองทรัพยากรกระดาษโดยใช่เหตุ นอกจากนี้ หลังจากเก็บข้อมูลลงกระดาษแล้ว ทางเนอสเซอรี่เองยังจําเป็นต้องกรอกข้อมูลลงในคอมพิวเตอร์อีกครั้งหนึ่ง 
    เพื่อป้องกันการสูญหายและอํานวยความสะดวกในการค้นหา
\enskip
ข้อมูล โครงงานนี้มุ่งที่จะแก้ไข ปัญหาและอุปสรรคดังกล่าว โดยการสร้างโปรแกรมประยุกต์บนเว็บ (web application) สําหรับจัดการข้อมูลภายเนอสเซอรี่ เพื่อเก็บข้อมูลโดยตรง ทําให้ลดความยุ่งยากในการจัดเก็บข้อมูลลดความผิดพลาดในการกรอกข้อมูลและยังสามารถนําข้อมูลที่มีอยู่ไปวิเคราะห์และแสดงผลเพื่อให้เห็นข้อมูลภาพรวมของเด็กได้ชัดเจนยิ่งขึ้น นอกจากนี้การใช้ระบบจัดการเนอสเซอรี่ยังช่วยลดการสูญหายของข้อมูลระหว่างการจัดเก็บ( การสูญหายของเอกสารที่มักพบในการเก็บข้อมูลในแบบเก่า, เอกสารเสียหาย )  ระบบจัดการเนอสเซอรี่นี้จะช่วยให้การบริหารจัดการเป็นไปอย่างคล่องตัวและเบ็ดเสร็จ ตั้งแต่กระบวนการรับเด็ก แรกเข้า การเลื่อนชั้น จนกระทั่งถึงเวลาที่เด็กพร้อมจะเข้าเรียนในระดับอนุบาล


\end{abstractTH}

\begin{abstract}
    Nursery management needs to store information to be used to monitor the readiness of each child each day, such as the statistics of coming-missing Personal equipment And health information Including emergency information such as the parents' emergency hospital contact number And personal information. At present, the collection of such information continues. Use the method to fill in the document. These mistakes easily take a long time and waste paper resources by accident. The nursery also needs to enter the information on the computer one more time.
    To prevent loss and facilitate searching.
    This project information is intended to be revised. Such problems and obstacles By creating a web application (web application) for managing data within the nursery. To collect information directly This simplifies data storage, reduces errors in filling out, and can also be used to analyze and display the existing data to give a clearer overview of the child. Using a nursery management system can also reduce data loss during storage (document loss often found in old-fashioned archives, damaged documents). Management is flexible and comprehensive. Since the process of adopting the first child into promotion Until the time the child is ready to enter kindergarten
Make sure your abstract sits inside the \texttt{abstract} environment.
\end{abstract}

\iffalse
\begin{dedication}
This document is dedicated to all Chiang Mai University students.

Dedication page is optional.
\end{dedication}
\fi % \iffalse

\begin{acknowledgments}
Your acknowledgments go here. Make sure it sits inside the
\texttt{acknowledgment} environment.

\acksign{2020}{5}{25}
\end{acknowledgments}%
\fi % \ifproject

\contentspage

\ifproject
\figurelistpage

\tablelistpage
\fi % \ifproject

% \abbrlist % this page is optional

% \symlist % this page is optional

% \preface % this section is optional
