\maketitle
\makesignature

\ifproject
\begin{abstractTH}
    การบริหารจัดการเนอสเซอรี่จําเป็นต้องจัดเก็บข้อมูลเพื่อใช้ในการตรวจสอบความพร้อมของเด็กในแต่ละวัน อาทิ สถิติการมา–ขาด อุปกรณ์ส่วนตัว และข้อมูลสุขภาพ รวมไปถึงข้อมูลยามฉุกเฉิน อาทิ หมายเลขติดต่อผู้ปกครอง โรงพยาบาลฉุกเฉิน และข้อมูลส่วนบุคคล ในปัจจุบัน การเก็บข้อมูลดังกล่าวยังคง ใช้วิธีกรอกลงบนเอกสาร ซึ่งเกิดความผิดพลาดได้ง่าย ใช้เวลานาน และสิ้นเปลืองทรัพยากรกระดาษโดยใช่เหตุ นอกจากนี้ หลังจากเก็บข้อมูลลงกระดาษแล้ว ทางเนอสเซอรี่เองยังจําเป็นต้องกรอกข้อมูลลงในคอมพิวเตอร์อีกครั้งหนึ่ง 
    เพื่อป้องกันการสูญหายและอํานวยความสะดวกในการค้นหาข้อมูล โครงงานนี้มุ่งที่จะแก้ไขปัญหาและอุปสรรคดังกล่าว โดยการสร้างโปรแกรมประยุกต์บนเว็บ (web application) สําหรับจัดการข้อมูลภายเนอสเซอรี่ เพื่อเก็บข้อมูลโดยตรง ทําให้ลดความยุ่งยากในการจัดเก็บข้อมูล ลดความผิดพลาดในการกรอกข้อมูล และยังสามารถนําข้อมูลที่มีอยู่ไปวิเคราะห์และแสดงผลเพื่อให้เห็นข้อมูลภาพรวมของเด็กได้ชัดเจนยิ่งขึ้น นอกจากนี้การใช้ระบบจัดการเนอสเซอรี่ยังช่วยลดการสูญหายของข้อมูลระหว่างการจัดเก็บ (เช่น การสูญหายของเอกสารที่มักพบในการเก็บข้อมูลในแบบเก่า และการเกิดความเสียหายของเอกสาร)  ระบบจัดการเนอสเซอรี่นี้จะช่วยให้การบริหารจัดการเป็นไปอย่างคล่องตัวและเบ็ดเสร็จ ตั้งแต่กระบวนการรับเด็กแรกเข้า การเลื่อนชั้น จนกระทั่งถึงเวลาที่เด็กพร้อมจะเข้าเรียนในระดับอนุบาล


\end{abstractTH}

\begin{abstract}
    Nursery management needs to store information to be used to monitor the readiness of each child each day, such as the attendance statistics, the list of personal equipment, and health information, including emergency information such as parents' emergency contact number and preferred hospital. At present, the collection of such information is still done by pen and paper.  This causes unnecessary paper consumption.  Moreoever, when mistakes happen, they take a long time to be corrected.  Also, to prevent loss and to facilitate convenient lookups, the nursery needs to reenter all the information on the computer anyway.
    The purpose of this project is to eliminate the unnecessary workflow by creating a web application for managing data within the nursery. Although outside the scope of this project, the collected information can be used afterwards for analyses and display to give a clearer perspective of the child. Using a nursery management system can also reduce data loss during storage (such as document losses and damages often found in old-fashioned archives).  This web application aims to be a one-stop service in managing a nursery, from the moment a child enters the nursery to departure for kindergarten.
    \CIreply{ใช้ Google Translate มาเหรอ?  Capitalization รวมทั้งเครื่องหมายต่างๆ ไม่เช็คเลย}
\end{abstract}

\iffalse
\begin{dedication}
This document is dedicated to all Chiang Mai University students.

Dedication page is optional.
\end{dedication}
\fi % \iffalse

\begin{acknowledgments}
โครงงานนี้สำเร็จลุล่วงได้ด้วยความกรุณาจากอาจารย์ชินวัตร อิศราดิสัยกุล อาจารย์ที่ปรึกษาโครงงานที่ได้ให้คำแนะนำและข้อเสนอแนะ แนวคิด ตลอดจนการค่อยเฝ้าติดตามและแก้ไขข้อบกพร่องต่างๆ ตลอดจนโครงงานเล่มนี้เสร็จสมบูรณ์
ผู้ศึกษาจึงขอกราบขอบพระคุณเป็นอย่างยิ่ง
ขอกราบขอบพระคุณคณะกรรมการสอบโครงงาน อาจารย์นวดนย์ คุณเลิศกิจ และ อาจารย์พฤษภ์ บุญมา ที่ค่อยให้คำแนะนำและข้อเสนอแนะต่างๆ ผู้ศึกษาจึงขอกราบขอบพระคุณเป็นอย่างยิ่ง
ขอบคุณเพื่อนๆที่ค่อยช่วยเหลือและให้คำแนะนำต่างๆ เกี่ยวกับโครงงานนี้
สุดท้ายนี้ขอขอบคูณทางพีคอะบู เนอร์สเซอรี่ ที่ได้ให้ความร่วมมือในการลองใช้งานระบบจัดการ nursery จนทำให้โครงงานสำเร็จลุล่วงไปได้ด้วยดี

\acksign{2020}{5}{25}
\CIreply{แก้วันที่}
\end{acknowledgments}%
\fi % \ifproject

\contentspage

\ifproject
\figurelistpage

\tablelistpage
\fi % \ifproject

% \abbrlist % this page is optional

% \symlist % this page is optional

% \preface % this section is optional
