\maketitle
\makesignature

\ifproject
\begin{abstractTH}
เนื่องในปัจุบันการจัดการข้อมูลภายใน Nursery ยังคงใช้วิธีการเก็บข้อมูลแบบเดิมๆ อยู่ซึ่งก็คือการจดลงในกระดาษก่อนแล้วค่อยนำมากรอกลงในคอมพิวเตอร์ ตัวอย่างของข้อมูล เช่น การเช็คชื่อนักเรียน ตรวจเช็คอุปกร์ของนักเรียน การกรอกใบสมัคร การตรวจสุขภาพ 
\enskip
ซึ่งการเก็บข้อมูลนั้นเก็บไปเพื่อใช้ในการตรวจสอบข้อมูลเด็กแต่ละคนและเก็บข้อมูลสำคัญ(เบอร์ติดต่อผู้ปกครอง โรงพยาบาลฉุกเฉิน ข้อมูลทางการแพทย์ของเด็กแต่ละคน)
\enskip
ซึ่งในการเก็บข้อมูลข้างต้นที่กล่าวมานั้นต้องเก็บโดยเขียนลงในกระดาษก่อนและค่อยนำไปกรอกใส่ในคอมพิวเตอร์ในภายหลังซึ่งทำให้เสียเวลาในการกรอกและสิ้นเปลืองทรัพยากรณ์กระดาษไปโดยใช่เหตุ 
\enskip 
จากปัญหาที่เกิดขึ้นทางผู้พัฒนาจึงคิดระบบจัดการภายในของ Nursery ขึ้นมาเพื่อลดความยุ่งยากในการเก็บข้อมูลลง ซึ่งระบบของเรานั้นการเก็บข้อมูลทำได้โดยการกรอกลงไปในคอมพิวเตอร์โดยผ่านโปรแกรมประยุกต์บนเว็บ(Web Application) ได้โดยตรง
\enskip
นอกจากการแก้ปัญหาข้างต้นแล้วยังมีประโยชน์ในเรื่องของการดูข้อมูลต่างๆ ได้จากภาย นอกที่ทำงาน กล่าวคือเราสามารถดูข้อมูลโดยผ่านเว็บไซต์ของเราได้ต่างจากการเก็บข้อมูลแบบเดิมที่สามารถดูได้เฉพาะที่ทำงานเท่านั้น จากเหตุผลข้างต้นที่กล่าวมาระบบของเราช่วยอำนวยความสะดวกต่อผู้ใช้งาน


\end{abstractTH}

\begin{abstract}
The abstract would be placed here. It usually does not exceed 350 words
long (not counting the heading), and must not take up more than one (1) page
(even if fewer than 350 words long).

Make sure your abstract sits inside the \texttt{abstract} environment.
\end{abstract}

\iffalse
\begin{dedication}
This document is dedicated to all Chiang Mai University students.

Dedication page is optional.
\end{dedication}
\fi % \iffalse

\begin{acknowledgments}
Your acknowledgments go here. Make sure it sits inside the
\texttt{acknowledgment} environment.

\acksign{2020}{5}{25}
\end{acknowledgments}%
\fi % \ifproject

\contentspage

\ifproject
\figurelistpage

\tablelistpage
\fi % \ifproject

% \abbrlist % this page is optional

% \symlist % this page is optional

% \preface % this section is optional
