\maketitle
\makesignature

\ifproject
\begin{abstractTH}
เนื่องในปัจุบันการจัดการภายใน Nusery ยังคงใช้\CI{วิธีการเก็บข้อมูลแบบเดิมๆ อยู่}{วิธีเดิมๆ คืออะไร ทำไมเดิมๆ}ทำให้เสียเวลาในการจัดเก็บข้อมูล เช่น การเช็คชื่อนักเรียน ตรวจเช็คอุปกร์ของนักเรียน การกรอกใบสมัคร ซึ่งในการเก็บข้อมูลข้างต้นที่กล่าวมานั้นต้องเก็บโดยใช้กระดาษในการเขียนและค่อยนำไปกรอกใส่ในคอมพิวเตอร์ที่หลังทำให้เสียเวลาในการกรอกทั้งสองรอบและสิ้นเปลืองทรัพยากรณ์อย่างกระดาษไปโดยเปล่าประโยชน์
\enskip 
ผู้พัฒนาจึงได้คิดค้นระบบจัดการภายในของ Nusery ขึ้นมาโดยทำเป็น  Web Appication \CI{โดยใช้ Famework เป็น ReactJs และ Express ของ NodeJs โดย Databas ใช้ mongoDB}{ละเอียดเกินไปใน abstract}
\CI{}{เรียงลำดับความคิดประมาณนี้
\begin{itemize}[nosep,leftmargin=*]
    \item nursery ในปัจจุบันเก็บข้อมูลอย่างไร เก็บอะไรบ้าง
    \item เก็บข้อมูลไปเพื่ออะไร
    \item ข้อเสียหรือปัญหาของการเก็บข้อมูลแบบที่มีอยู่คืออะไร
    \item เพื่อลดข้อเสียหรือปัญหา โครงงานนี้จึงทำอะไร
    \item ลดข้อเสียหรือปัญหาอย่างไรได้บ้าง
    \item นอกจากลดข้อเสียหรือปัญหาแล้ว ที่เราจะทำมีประโยชน์อย่างอื่นนอกเหนือจากระบบเดิมหรือไม่
    \item (พูดถึงการประเมินระบบ?)
\end{itemize}
}
\end{abstractTH}

\begin{abstract}
The abstract would be placed here. It usually does not exceed 350 words
long (not counting the heading), and must not take up more than one (1) page
(even if fewer than 350 words long).

Make sure your abstract sits inside the \texttt{abstract} environment.
\end{abstract}

\iffalse
\begin{dedication}
This document is dedicated to all Chiang Mai University students.

Dedication page is optional.
\end{dedication}
\fi % \iffalse

\begin{acknowledgments}
Your acknowledgments go here. Make sure it sits inside the
\texttt{acknowledgment} environment.

\acksign{2020}{5}{25}
\end{acknowledgments}%
\fi % \ifproject

\contentspage

\ifproject
\figurelistpage

\tablelistpage
\fi % \ifproject

% \abbrlist % this page is optional

% \symlist % this page is optional

% \preface % this section is optional
