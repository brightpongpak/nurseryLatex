\chapter{\ifcpe บทสรุปและข้อเสนอแนะ\else Conclusions and Discussion\fi}

\section{\ifcpe สรุปผล\else Conclusions\fi}

ตัวโครงงาน  Nursery Management System  ทางผู้พัฒนาจัดทำขึ้นเพื่อปรับเปลี่ยนจากการจัดเก็บข้อมูลแบบเดิม(เอกสาร)ไปเป็นการจัดเก็บข้อมูลแบบใหม่ลงบน Plantform ออนไลน์ที่จะช่วยลดขั้นตอนในการจัดส่งข้อมูลและลดเวลาในการจัดการกับเอกสารต่างๆ โดยการทำผ่านตัว Application ในส่วนข้อจำกัดของระบบคือ ทาง Application ของเราสามารถทำได้หลักๆเพียง 7 Features คือ หน้าจัดการเกี่ยวประวัติของเด็ก (Profile), หน้าลงทะเบียนเด็ก (Register), หน้าเช็คชื่อเด็ก(Attendance), หน้าเช็คของเด็ก (Gadget),หน้าเช็คสุขภาพ (Health), หน้าจัดการ Stock ของ (Stock), หน้า Payment ทางระบบของเราจะยังไม่รองรับฟังก์ชั่นการทำงานอื่นๆนอกเหนือจากนี้

\section{\ifcpe ปัญหาที่พบและแนวทางการแก้ไข\else Challenges\fi}

ปัญหาส่วนใหญ่ที่พบก็คือ การออกแบบที่ยังไม่ตรงตามความต้องการของผู้ใช้จริง  การสื่อสารกันที่ไม่ชัดเจน ทำให้ความเข้าใจของทั้งสองฝั่งต่างกัน จึงทำให้เกิดผลลัพธ์ที่ไม่ต้องการ  สาเหตุหลักๆก็มาจากการที่ผู้พัฒนายังอ่อนประสบการณ์ในสายงานพัฒนาเว็บ application และยังขาดทักษะในด้านการสื่อสารที่ยังไม่ชัดเจนเท่าที่ควร ทำให้ชิ้นงานยังเกิดข้อผิดพลาดอยู่
ในส่วนของแนวทางการแก้ไขปัญหาก็คือ ทางผู้พัฒนาต้องให้เวลาในส่วนของการออกแบบระบบและมีการไปพูดคุยกับทางผู้ใช้ให้เคลียร์ เพื่อให้ตอบโจทย์ลูกค้าให้มากที่สุด และ ออกแบบฐานข้อมูลให้สามารถรองรับข้อมูลภายในหน้า UI ได้อย่างครอบคลุมทั้งหมดเพื่อลดการกลับมาแก้ไขโครงสร้างฐานข้อมูลในภายหลัง

\section{\ifcpe%
ข้อเสนอแนะและแนวทางการพัฒนาต่อ
\else%
Suggestions and further improvements
\fi
}

ถ้าหากทาง Nursery ต้องการนำตัว plantform นี้ไปใช้งานต่อจริง ทางผู้พัฒนาก็จะบอกคุณสมบัติและข้อจำกัดทั้งหมดของตัวเว็บนี้ว่าสามารถทำอะไรได้บ้าง ส่วนไหนไม่สามารถทำได้
ส่วนในแนวทางการพัฒนาต่อ  ทางผู้พัฒนาคิดจะพัฒนาจากระบบ Client-Server ไปเป็นระบบแบบ Microservice เพื่อให้สามารถจัดการกับฟังก์ชั่นการทำงานต่างๆ ที่เพิ่มเข้ามาใหม่ให้สามารถทำงานควบคู่ไปกับตัวฟีเจอร์ก่อนหน้านี้ได้อย่างเป็นระบบระเบียบ

