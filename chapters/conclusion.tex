\chapter{\ifcpe บทสรุปและข้อเสนอแนะ\else Conclusions and Discussion\fi}

\section{\ifcpe สรุปผล\else Conclusions\fi}

ตัวโครงงาน  Nursery Management System  ทางผู้พัฒนาจัดทำขึ้นเพื่อปรับเปลี่ยนจากการจัดเก็บข้อมูลแบบเดิม (เอกสาร) ไปเป็นการจัดเก็บข้อมูลแบบใหม่ลงบน platform ออนไลน์ที่จะช่วยลดขั้นตอนในการจัดส่งข้อมูลและลดเวลาในการจัดการกับเอกสารต่างๆ ผ่าน web application 

ในส่วนข้อจำกัดของระบบคือ ทาง application ของเราสามารถทำได้หลักๆ เพียง 7 features ได้แก่ หน้าจัดการเกี่ยวประวัติของเด็ก (profile), หน้าลงทะเบียนเด็ก (register), หน้าเช็คชื่อเด็ก (attendance), หน้าเช็คของเด็ก (gadget), หน้าเช็คสุขภาพ (health), หน้าจัดการสต็อกสินค้า (stock), หน้า payment ทางระบบของเราจะยังไม่รองรับฟังก์ชันการทำงานอื่นๆ นอกเหนือจากนี้

\section{\ifcpe ปัญหาที่พบและแนวทางการแก้ไข\else Challenges\fi}

ปัญหาส่วนใหญ่ที่พบก็คือ การออกแบบที่ยังไม่ตรงตามความต้องการของผู้ใช้จริง  การสื่อสารกันที่ไม่ชัดเจน ทำให้ความเข้าใจของทั้งสองฝั่งต่างกัน เช่น ผู้พัฒนาคิดว่า client ต้องการแค่หน้าเช็คชื่อ
แต่ client กลับต้องการหน้าปริ้น form เช็คชื่อด้วย จึงทำให้เกิดผลลัพธ์ที่ไม่ต้องการ  

สาเหตุหลักๆ ก็มาจากการที่ผู้พัฒนายังอ่อนประสบการณ์ในสายงานพัฒนา web application และยังขาดทักษะในด้านการสื่อสารที่ยังไม่ชัดเจนเท่าที่ควร ทำให้ชิ้นงานยังเกิดข้อผิดพลาดอยู่
ในส่วนของแนวทางการแก้ไขปัญหาก็คือ ทางผู้พัฒนาต้องให้เวลาในส่วนของการออกแบบระบบและมีการไปพูดคุยกับทางผู้ใช้ให้เรียบร้อย เพื่อให้ตอบโจทย์ลูกค้าให้มากที่สุด และ ออกแบบฐานข้อมูลให้สามารถรองรับข้อมูลภายในหน้า UI ได้อย่างครอบคลุมทั้งหมดเพื่อลดการกลับมาแก้ไขโครงสร้างฐานข้อมูลในภายหลัง
\section{\ifcpe%
ข้อเสนอแนะและแนวทางการพัฒนาต่อ
\else%
Suggestions and further improvements
\fi
}

ถ้าหากทาง nursery ต้องการนำตัว platform นี้ไปใช้งานต่อจริง ทางผู้พัฒนาจะบอกคุณสมบัติและข้อจำกัดทั้งหมดของตัวเว็บนี้ว่าสามารถทำอะไรได้บ้าง ฟังก์ชันการทำงานไหนไม่สามารถทำได้
ส่วนในแนวทางการพัฒนาต่อ  ทางผู้พัฒนาคิดจะพัฒนาจากระบบ client--server ไปเป็นระบบแบบ microservice เพื่อให้สามารถจัดการกับฟังก์ชันการทำงานต่างๆ ที่เพิ่มเข้ามาใหม่ให้สามารถทำงานควบคู่ไปกับตัวฟีเจอร์ก่อนหน้านี้ได้อย่างเป็นระบบระเบียบ
